%%ultima modificacion 15-09-99
%\input format
%\input extdef

%%%%%%%%%%%%%%%%%%%%%%%%%%%%%%%%%%%%%%%%%%%%%%%%%%%%%%%%%%%%%%%%%%%
%
% Problemas sobre el teorema de existencia y unicidad de la ecuacion
% de una sola variable
%
%%%%%%%%%%%%%%%%%%%%%%%%%%%%%%%%%%%%%%%%%%%%%%%%%%%%%%%%%%%%%%%%%%%
%

\Pro 1.- [Kiseliov]
Encuentre todas las soluciones de 
\begin{equation}
  \label{eq:tresmedios}
  \frac{dx}{dt} = \frac{3}{2} x^{3/2}.
\end{equation}
%
Ayuda: hay infinitas y estas se obtienen a partir de segmentos de algunas
soluciones particulares.
Grafique algunas de estas soluciones.
\ePro

\Pro 2.- [Kiseliov]
Aplique el teorema de existencia y unicidad para determinar las regiones
donde las ecuaciones siguientes tienen soluci\'on \'unica:

a) $\dot{x} = x + 3 x^{1/3}$.

b) $\dot{x} = 1 -\ctg x$.

c) $\dot{x} = \sqrt{1-x^2}$.
\ePro

\Pro 3.- [Kiseliov]
Resuelva las siguientes ecuaciones, en todos los casos de las
soluciones generales en funcio\'on de un dato inicial $x_0$
para el tiempo inicial $t_0$.

a) $t \dot{x} + x = cos(t)$. (Use integral primera de la parte homog\'enea.)

b) $\dot{x} + 2x = e^t$. (Use variaci'on de constantes.)

c) $(1-t^2)\dot{x} + tx = 2t$. (Resuelva primero la homog\'enea y luego
sume una inhomog\'enea particular.)

d) $x \dot{x} = t - 2t^3$.
\ePro

\Pro 4.- [Kiseliov]
Resuelva las siguientes ecuaciones apelando a un cambio de variable en
la variable independiente (t).

a) $\dot{x}x = -t$. ($t = \cos(s)$.)

b) $\dot{x}t - x =0$.

c) $\dot{x} + e^{\dot{x}} = t$.
\ePro

\Pro 5.- [Kiseliov]
Grafique las isoclinas (l\'\i{}neas de igual pendiente en el plano $(t,x)$)
y luego trace las soluciones de las siguientes ecuaciones:

a) $\dot{x} = 2t - x$.

b) $\dot{x} = \sin(x+t)$.

c) $\dot{x} = x -t^2 + 2t -2$.

d) $\dot{x} = \frac{x-t}{x+t}$.
\ePro

\Pro 6.- 
Resuelva la ecuaci\'on:

$\dot{x} = A(t)x + B(t)x^n$ Ayuda: Use el cambio de variable $y = x^{1-n}$.

\Pro 7.-
Resuelva la ecuaci\'on
\begin{equation}
  \label{eq:osc}
  \frac{dx}{dt} = i\lamdba x + A e^{i \omaga t} \;\; \lambda, \; \omega \in \re
\end{equation}
%
y vea como se comporta su parte real.
Examine los casos: 
a) $A=0$, 
b) ($A \neq 0, \;\; \lambda \neq \omega$) y
c) ($A \neq 0, \;\; \lambda = \omega$).

\Pro 8.-
La ecuaci\'on del p\'endula f\'\i{}sico.

a) Grafique el campo vectorial de la ecuaci\'on
\begin{equation}
  \label{eq:pendulo_fisico}
  \frac{d^2\theta}{dt^2} = -\sin(\theta), \;\;\;-\pi \leq \theta \leq \pi. 
\end{equation}

b) Grafique algunas curvas integrales alrededor de 
$(\theta=0, z=\frac{d\theta}{dt} = 0)$.

c) Grafique las curvas integrales que pasan por el punto
$\theta = \pm \pi, z=0)$. Infiera que el tiempo necesario 
para llegar por estas curvas a dicho punto es infinito.

d) Grafique el campo vectorial a lo largo de una recta $z=z_0$.
Infiera de ello que las soluciones no pueden superar nunca la velocida adquirida cuando pasan por el punto $\theta=0$. Ayuda: Concluya esto primero para la
regi'on $0 \leq \theta \leq \pi,\;\;\; 0 \leq z \leq z_0$ y luego use la simetr\'ia de la soluci\'on.

c) Sea $E(z,\theta) := \frac{z^2}{2} - \cos(\theta)$. Vea que 
$\frac{dE}{dt} = 0$. Use esta cantidad para analizar el comportamiento
de las soluciones con dato inicial $(z_0,\theta_0=0)$. En particular
determine cuales cruzan el eje $z=0$ y cuales no.
\ePro

\Pro 9.-
Sea la ecuaci\'on:
\begin{equation}
  \frac{dz}{dt} = \left\{
  \begin{array}{ll}
    f(z) & |z| < 1 \\
    iz   & |z| \geq 1
  \end{array}
  \right.
\end{equation}
Donde $f(z)$ es cont\'\i{}nua con $iz$ en $|z|=1$.
Infiera que no importando como sea la forma de $f(z)$, las soluciones
nunca escapan de la regi\'on $\{|z(t)| \leq \max\{|z(0)|, 1\}\}$.
\ePro

\Pro 10.-
Sea un cuerpo afectado por una fuerza central, es decir,
\begin{equation}
  m\frac{d^2\vec{x}}{dt^2} = f(r)\vec{x}\;\;\; r:= |\vcec{x}|.
\end{equation}
%
a) Encuentre un sistema equivalente de primer orden.

b) Compruebe que dado un vector cualquiera (constante) $\vec{c}$, luego
   $F(\vec{x},\vec{p}) := \vec{c}\cdot(\vec{x}\wedge \vec{p})$,
   donde $\vec{p} := \frac{d\vec{x}}{dt}$, es una integral primera.
\ePro

\Pro 11.-
Sea la ecuaci\'on:
\begin{equation}
  \frac{d\vec{x}}{dt} = \vec{x} \wedge \vec{\omega}(\vec{x}).
\end{equation}
%
a) Vea que $R:= \vec{x}\cdot\vec{x}$ es una integral primera.

b) Concluya que las soluciones de este sistema existen para todo tiempo.
\ePro















 
%\end

%%% Local Variables: 
%%% mode: latex
%%% TeX-master: "apu_tot"
%%% End: 
