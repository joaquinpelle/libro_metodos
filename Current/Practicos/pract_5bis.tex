\bpro[La ecuaci\'on de Volterra-Lotka]

Sea el sistema:

\begin{eqnarray}
  \label{eq:Volterra_Lotka}
  \dot{x}_1 &=& (a-bx_2)x_1 \nn
  \dot{x}_2 &=& -(c-f x_1)x_2,
\end{eqnarray}
$a, b, c, f \geq 0$.

a) Haga una trasformaci\'on de coordenadas y tiempo que lo lleve a la forma,
\begin{eqnarray}
  \label{eq:Volterra_Lotka_adim}
  \dot{x}_1 &=& (1 - x_2)x_1 \nn
  \dot{x}_2 &=& -(e - x_1)x_2
\end{eqnarray}
%

b) Grafique el campo vectorial y vea que no hay integrales primeras no
triviales en los cuadrantes donde al menos una de las coordenadas es
negativa.

c) Encuentre las soluciones de equilibrio y determine cuales son 
estables y cuales no.

d) Examine el cuadrante positivo mediante la transformaci\'on:
$x_1 = e^{q_1}$, $x_2 = e^{q_2}$ y vea que alli la cantidad
$f(q_1,q_2) := q_1 + q_2 -(e^{q_1} + e^{q_2})$ es una integral
de movimiento. Utilize esta informaci\'on para inferir que en ese
cuadrante las trayectorias permanecen en regiones acotadas.

Nota, esta ecuaci\'on describe la poblaci\'on de dos especies en competencia.
Lo que acabamos de ver es que las especies no crecen indefinidamente ni
desaparecen. Esto \'ultimo nos dice que la aproximaci\'on no es
muy buena...
\epro

\bpro[Ciclo l\'imite]
Considere el sistema:
\begin{eqnarray}
  \label{eq:Ciclo_Limite_Prob}
  \dot{x}_1 &=& x_2 + x_1(1-(x_1^2+x_2^2)) \nn
  \dot{x}_2 &=& -x_1 + x_2(1-(x_1^2+x_2^2)).
\end{eqnarray}
%

a) Lleve este sistema a un par de ecuaciones desacopladas,
\begin{eqnarray}
  \label{eq:Ciclo_Limite_Prob_des}
  \dot{r} &=&  f(r) \nn
  \dot{\theta} &=& -1.
\end{eqnarray}
%

b) Estudie los puntos de equilibrio de la primera ecuaci\'on y su estabilidad.
?'Cual es la soluci\'on del sistema original correspondiente a este punto
de equilibrio?


c) Grafique soluciones cerca de estos puntos, en el plano $(r,\theta)$ y en el
plano $(x_1,x_2)$.






%%% Local Variables: 
%%% mode: latex
%%% TeX-master: "apu_tot"
%%% End: 
