



\bpro
Dado un conjunto de funciones $\{f_i(t)\}, \;i=1,..,n$ se definen vectores
$\ve{u}_i(t) := (f_i(t), f^{(1)}_i(t),\dotsf^{(n-1)})$ y el
Wronskiano del sistema como 
$W(\{f_i\})(t) := \eps(\ve{u}_1(t),\ve{u}_2(t),\dots, \ve{u}_n(t))$.
Si el Wronskiano de un conjunto de funciones no se anula, entonces las
funciones son linealmente independientes, es decir ninguna combinaci\'on 
lineal no trivial de las mismas (con coeficientes constantes) se anula.
La conversa no es cierta.
Calcule el Wronskiano de los siguientes conjuntos:

a) $\{4,t\}$

b) $\{t,3t,t^2\}$

c) $\{e^t, te^t, t^2e^t\}$

d) $\{ \sin(t), \cos(t), \cos(2t) \}$

e) $\{1, \sin(t), \sin(2t)\}$
\epro

\bpro
Decida si el siguiente conjunto de funciones es linealmente depediente o
no. Luego calcule el Wronskiano.

\begin{equation}
  f_1(t) = \left\{
    \begin{array}{ll}
      0, & 0 \leq x \leq 1/2 \\
      (x-1/2)^2, & 1/2 \leq x \leq 1
    \end{array}
    \right.
\end{equation}

\begin{equation}
  f_2(t) = \left\{
    \begin{array}{ll}
      (x-1/2)^2, & 0 \leq x \leq 1/2 \\
      0, & 1/2 \leq x \leq 1
    \end{array}
    \right.
\end{equation}
\epro

\bpro
Lleve a forma de primer orden los siguientes sistemas y calcule sus
soluciones generales. Grafique alguna de ellas.

a) 
\begin{equation}
  \label{eq:prob5_2a}
  \frac{d}{dt}\left(
    \begin{array}{c}
      x_1 \\ x_2
    \end{array}
  \right) = 
    \left(
      \begin{array}{cc}
        1 & 0 \\ 0 & 1
      \end{array}
    \right) \left(
      \begin{array}{c}
        x_1 \\ x_2
      \end{array}
      \right)
\end{equation}

\epro

\bpro
Hallar la soluci\'on general de las ecuaciones:

a) $\frac{d^3 x}{dt^3} - 2\frac{d^2 x}{dt^2} - 3\frac{d x}{dt} =0$

b) $\frac{d^3 x}{dt^3} + 2\frac{d^2 x}{dt^2} + \frac{d x}{dt} =0$

c) $\frac{d^3 x}{dt^3} + 4\frac{d^2 x}{dt^2} + 13 \frac{d x}{dt} =0$

d) $\frac{d^4 x}{dt^4} + 4\frac{d^3 x}{dt^3} + 8\frac{d^2 x}{dt^2} + 8 \frac{d x}{dt} + 4 = 0$
\epro

%%% Local Variables: 
%%% mode: latex
%%% TeX-master: "apu_tot"
%%% End: 
