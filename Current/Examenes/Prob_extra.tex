\documentclass{article}
%\input{extdef}
\newcommand{\ve}[1]{\mbox{\boldmath ${#1}$}}
\begin{document}

\begin{center}
 \textbf{Problemas de examen  1999 }
\end{center}


\begin{center}
 \textbf{Problemas de ecuaciones (Kiseliov)}
\end{center}
\vspace{1cm}

Problema 1.- Encontrar todas las soluciones de 
\begin{equation}
t^2 y y'' = (y-ty')^2
\end{equation}
%
Ayuda: Plantee una soluci\'on del tipo $y(t) = e^{z(t)}$.

Problema 2.- Encuentre la soluci\'on general de la ecuaci\'on:
\begin{equation}
  x^3 y'' = (y-xy')^2
\end{equation}
%
Ayuda: Plantee una soluci\'on del tipo $y(t) = u(t)x$, con $t$ dado por
$x=e^t$. Reduzca luego el problema a uno de primer orden.

Problema 3.- Encuentre la soluci\'on general de la ecuaci\'on:
\begin{equation}
  t y''' - y'' = 0
\end{equation}
\vspace{1cm}

\begin{center}
 \textbf{Sistemas Lineales (Kiseliov)}
\end{center}


Problema 1.- Lleve a primer orden y resuelva:

\begin{equation}
y''' - y'' + y' -y = t^2 + t
\end{equation}


Problema 2.- Lleve a primer orden y resuelva:

\begin{equation}
y''' - y'' = 12t^2 + 6t
\end{equation}

Problema 3.- Lleve a primer orden y resuelva:

\begin{equation}
y'' - 6y' + 9y = 25e^t \sin(t)
\end{equation}

\newpage

\begin{center}
 \textbf{Sistemas, Estabilidad (Kiseliov)}
\end{center}
\vspace{1cm}

Problema 1.- Estudie la estabilidad de la soluci\'on $(x_1=0,x_2=0)$ del sistema:

\begin{eqnarray}
  \dot{x}_1 &=& \frac{3}{4}\sin{x_1} - 7x_2(1-x_2)^{\frac{1}{3}} + x_1^3 \nonumber \\
  \dot{x}_2 &=& \frac{2}{3}x_1 - 3x_2 \cos(x_2) -11x_2^5
\end{eqnarray}

Problema 2.- Estudie la estabilidad de la soluci\'on $(x_1=0,x_2=0)$ del sistema:

\begin{eqnarray}
  \dot{x}_1 &=& \frac{1}{4}(e^{x_1}-1) -9x_2 + x_1^4 \nonumber \\
  \dot{x}_2 &=& \frac{1}{5}x_1 - \sin(x_2) + x_2^{14}
\end{eqnarray}

Problema 3.- Estudie la estabilidad de la soluci\'on $(x_1=0,x_2=0)$ del sistema:

\begin{eqnarray}
  \dot{x}_1 &=& 5x_1 + x_2 \cos(x_2) - \frac{x_1^3}{3} \nonumber \\
  \dot{x}_2 &=& 3x_1 + 2x_2 + \frac{x_1^4}{12}  - x_2e^{x_2}
\end{eqnarray}

\newpage

\begin{center}
 \textbf{Algebra, Diagonalizaci\'on}
\end{center}
\vspace{1cm}

Problema 1.- Diagonalize la siguiente matriz y escriba las componentes
del vector dado con respecto a la base en la cual la matriz es diagonal.

\begin{equation}
  \frac{1}{4}\left( 
    \begin{array}{ccc}
         3 & 2 & 1 \nonumber \\
         1 & 2 & -1 \nonumber \\
         -1 & -2 & 1 
    \end{array}
    \right)
    \;\;\;\;\;
    \left( 
    \begin{array}{c}
          1  \nonumber \\
          3  \nonumber \\
          2  
    \end{array}
    \right)
\end{equation}


Problema 2.- Diagonalize la siguiente matriz y escriba las componentes
del vector dado con respecto a la base en la cual la matriz es diagonal.

\begin{equation}
  \left( 
    \begin{array}{ccc}
         3 & 4 &  3 \\
         1 & 0 & -3 \\
         -1 & -4 &  -1
    \end{array}
    \right)
    \;\;\;\;\;
    \left( 
    \begin{array}{c}
          1  \\
          3  \\
          2  
    \end{array}
    \right)
\end{equation}


Problema 3.- Diagonalize la siguiente matriz y escriba las componentes
del vector dado con respecto a la base en la cual la matriz es diagonal.

\begin{equation}
  \left( 
    \begin{array}{ccc}
          3 & 3 & 2 \\
          1 & 1 & -2 \\
          -1 & -3 & 0  
    \end{array}
    \right)
    \;\;\;\;\;
    \left( 
    \begin{array}{c}
          1  \\
          3  \\
          2  
    \end{array}
    \right)
\end{equation}
\vspace{1cm}


\begin{center}
  \textbf{Serie de Fourier}
\end{center}
\vspace{1cm}

Problema 1.- Calcule la serie de Fourier en el intervalo $[0,2\pi]$ 
correspondiente a la siguiente funci\'on:

\begin{equation}
  f(x) := \sin(\alpha x) \;\;\;\; \alpha \mbox{irracional}
\end{equation}


Problema 2.- Calcule la serie de Fourier en el intervalo $[0,2\pi]$ 
correspondiente a la siguiente funci\'on:

\begin{equation}
XXX 
\end{equation}



Problema 3.- Calcule la serie de Fourier en el intervalo $[0,2\pi]$ 
correspondiente a la siguiente funci\'on:

\begin{equation}
 XXX
\end{equation}


\newpage

\begin{center}
  \textbf{Operadores Autoadjuntos y Unitarios}
\end{center}

Problema 1.- Sea $\ve{A} : V \to V$ un operador autoadjunto, probar que
$e^{i\ve{A}}$ es un operador unitario.

Problema 2.- Sea $\ve{U} : V \to V$ un operador unitario, probar:

a) Si $f:{\cal{L}}(V,V) \to R$ es una funci\'on real anal\'\i{}tica
(es decir, admite una representaci\'on como una serie de potencias en las
componentes de los elementos de ${\cal{L}}(V,V)$ con coeficientes reales).
Luego $f(\ve{A}^{\star}) = \bar{f}(\ve{A})$ para todo 
$\ve{A} \in {\cal{L}}(V,V)$.

b) $|\det(\ve{U})| = 1$.

Problema 3.- Probar que los autovectores correspondintes a distintos autovalores de un operador unitario son ortogonales.


\vspace{1cm}
%\newpage


\begin{center}
  \textbf{Verdadero / Falso: ODEs, Sistemas Lineales}
\end{center}

Problema 1.- El sistema 
\begin{equation}
\frac{d^3x}{dt^3} + \sin(x)\frac{dx}{dt} = \cos(\omega t), \;\;\;\;
              \omega \mbox{racional},
\end{equation}
%
tiene tres soluciones linealmente independientes.

Problema 2.- El Wronskiano del sistema,
\begin{equation}
  \frac{d^4x}{dt^4} + cosh(x) \frac{dx^2}{dt^2} = senh(x),
\end{equation}
%
se anula solo en un punto.

Problema 3.- Sea $X^t_{t_0}: V \to V$ el mapa que envia un vector $\ve{x}_0$
en el vector $\ve{x}(t)$ soluci\'on de la ecuaci\'on,
\begin{equation}
  \frac{d\ve{x}}{dt} = A(t)\ve{x} + \ve{b}(t),
\end{equation}
con $A(t)$ diferenciable y $\ve{b}(t)$ acotada, con condici\'on inicial
$\ve{x}(t_0) = \ve{x}_0$.
Luego, $(X^t_{t_0})^{-1}$ existe.

\newpage

\begin{center}
  \textbf{Verdadero / Falso: ODEs, Unicidad}
\end{center}
\vspace{1cm}

Problema 1.- Las ecuaciones siguientes tienen soluci\'on \'unica:

a) $\frac{d\ve{x}}{dt} = \frac{\sin(x)}{x}$.

b) $\frac{d\ve{x}}{dt} = x^{\frac{3}{2}}$.

c) $\frac{d\ve{x}}{dt} = x^{\frac{1}{2}} + cos(x)$.

Problema 2.- Las ecuaciones siguientes tienen soluci\'on \'unica:

a) $\frac{d\ve{x}}{dt} = \frac{1 - \cos(x)}{x}$.

b) $\frac{d\ve{x}}{dt} = e^{-\frac{1}{x^2}}$.

c) $\frac{d\ve{x}}{dt} = x^{\frac{5}{4}}$.


Problema 3.- Las ecuaciones siguientes tienen soluci\'on \'unica:

a) $\frac{d\ve{x}}{dt} = x\sin(1/x)$.

b) $\frac{d\ve{x}}{dt} = e^{-\frac{1}{x^2}}$.

c) $\frac{d\ve{x}}{dt} = (\sin(x))^{1/2}$.


%\newpage

\begin{center}
  \textbf{Verdadero / Falso: Fourier}
\end{center}

Problema 1.- Es necesario que una funci\'on sea diferenciable para que su
serie de Fourier converja punto a punto y uniformente.

Problema 2.- Existe una sucesi\'on de funciones $\{f_n(x)\}$ en $L^2$ tal 
que 
%$\|f_n\|_{L^2} = 1$ 
y tal que la norma en $l^2$ de sus coeficientes de
Fourier tiende a cero.

Problema 3.- El espacio $L^2(R)$ tiene una base ortonormal numerable.




\end{document}





