\documentclass{article}

\input{../Current/extdef}
\begin{document}

\begin{center}
  \textbf{Serie de Fourier}
\end{center}

Problema 1.- Calcule la serie de Fourier en el intervalo $[0,2\pi]$ 
correspondiente a la siguiente funci\'on:

\begin{equation}
  f(x) := \sin(\alpha x) \;\;\;\; \alpha \mbox{irracional}
\end{equation}


Problema 2.- Calcule la serie de Fourier en el intervalo $[0,2\pi]$ 
correspondiente a la siguiente funci\'on:

\begin{equation}
 
\end{equation}



Problema 3.- Calcule la serie de Fourier en el intervalo $[0,2\pi]$ 
correspondiente a la siguiente funci\'on:

\begin{equation}
 
\end{equation}


\newpage

\begin{center}
  \textbf{Operadores Autoadjuntos y Unitarios}
\end{center}

Problema 1.- Sea $\ve{A} : V \to V$ un operador autoadjunto, probar que
$e^{i\ve{A}}$ es un operador unitario.

Problema 2.- Sea $\ve{U} : V \to V$ un operador unitario, probar:

a) Si $f:{\cal{L}}(V,V) \to \re$ es una funci\'on real anal\'\i{}tica
(es decir, admite una representaci\'on como una serie de potencias en las
componentes de los elementos de ${\cal{L}}(V,V)$ con coeficientes reales).
Luego $f(\ve{A}^{\star}) = \bar{f}(\ve{A})$ para todo 
$\ve{A} \in {\cal{L}}(V,V)$.

b) $|\det(\ve{U})| = 1$.

Problema 3.- Probar que los autovectores correspondintes a distintos autovalores de un operador unitario son ortogonales.



\newpage


\begin{center}
  \textbf{Verdadero / Falso: ODEs, Sistemas Lineales}
\end{center}

Problema 1.- El sistema 
\begin{equation}
\\frac{d^3x}{dt^3} + \sin(x)\frac{dx}{dt} = \cos(\omega t), \;\;\;\;
              \omega \mbox{racional},
\end{equation}
%
tiene tres soluciones linealmente independientes.

Problema 2.- El Wronskiano del sistema,
\begin{equation}
  \frac{d^4x}{dt^4} + cosh(x) \frac{dx^2}{dt^2} = senh(x),
\end{equation}
%
se anula solo en un punto.

Problema 3.- Sea $X^t_{t_0}: V \to V$ el mapa que envia un vector $\ve{x}_0$
en el vector $\ve{x}(t)$ soluci\'on de la ecuaci\'on,
\begin{equation}
  \frac{d\ve{x}}{dt} = A(t)\ve{x} + \ve{b}(t),
\end{equation}
con $A(t)$ diferenciable y $\ve{b}(t)$ acotada, con condici\'on inicial
$\ve{x}(t_0) = \ve{x}_0$.
Luego, $(X^t_{t_0})^{-1}$ existe.

\newpage

\begin{center}
  \textbf{Verdadero / Falso: ODEs, Unicidad}
\end{center}

Problema 1.- Las ecuaciones siguientes tienen soluci\'on \'unica:

a) $\frac{d\ve{x}}{dt} = \frac{\sin(x)}{x}$.

b) $\frac{d\ve{x}}{dt} = x^{\frac{3}{2}$.

c) $\frac{d\ve{x}}{dt} = x^{\frac{1}{2} + 2$.

Problema 2.- Las ecuaciones siguientes tienen soluci\'on \'unica:

a) $\frac{d\ve{x}}{dt} = \frac{1 - \cos(x)}{x}$.

b) $\frac{d\ve{x}}{dt} = e^{-\frac{1}{x^2}}$.

c) $\frac{d\ve{x}}{dt} = x^{\frac{5}{4}$.


Problema 3.- Las ecuaciones siguientes tienen soluci\'on \'unica:

a) $\frac{d\ve{x}}{dt} = x\sin(1/x)$.

b) $\frac{d\ve{x}}{dt} = e^{-\frac{1}{x^2}}$.

c) $\frac{d\ve{x}}{dt} = (\sin(x))^{1/2}$.


\newpage

\begin{center}
  \textbf{Verdadero / Falso: Fourier}
\end{center}

Problema 1.- Es necesario que una funci\'on sea diferenciable para que su
serie de Fourier converja punto a punto y uniformente.

Problema 2.- Existe una sucesi\'on de funciones $\{f_n(x)\}$ en $L^2$ tal 
que $\|f_n\|_{L^2} = 1$ y tal que la norma en $l^2$ de sus coeficientes de
Fourier tiende a cero.

Problema 3.- El espacio $L^2(\re)$ tiene una base ortonormal numerable.








\end{document}

%%% Local Variables: 
%%% mode: latex
%%% TeX-master: t
%%% End: 
