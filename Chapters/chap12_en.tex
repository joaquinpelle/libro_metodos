% filepath: /Users/reula/Docencia/Metodos_libro/libro_metodos_github/translate.tex
% !TEX encoding = IsoLatin
% !TEX root =  ../Current_garamond/libro_gar.tex

%% last modification 28/05/2013

\chapter{Elliptic Equations}
\label{elliptic_equations}

\section{The Laplace Equation}

We will take the Laplace equation in $\re^n$ with Euclidean metric $ds^2 = \sum^n_{i=1} (dx^i)^2$,
or an open region $\Omega$ of $\re^n$ with the same metric, as a model of an elliptic equation.
The results we obtain that do not depend on special functions can be generalized to elliptic equations of the form \ron{ec*} if it is assumed that $c \leq 0$.

As we mentioned before, the Cauchy program, that is, formulating the problem of obtaining solutions by giving $u^A$ on a hypersurface $\Sigma$, does not generally work, it only works for hyperbolic equations.
If we consider non-analytic data, and physical data are generally non-analytic, there are generally not even local solutions.
What is the appropriate way then to give data for the Laplace equation? We will obtain the answer to this by considering a physical phenomenon described by this equation.

Consider a drum and apply a (small) force on its membrane (or patch) perpendicular to it.

\espa 
%\fig{6cm}{The Drum Membrane.}
\begin{figure}[htbp]
  \begin{center}
    \resizebox{7cm}{!}{\myinput{Figure/m12_1}}
    \caption{The drum membrane.}
    \label{fig:12_1}
  \end{center}
\end{figure}

The membrane will move from its flat (rest) position and acquire a new equilibrium position generating an elastic force that exactly cancels the applied one.
What will this new shape of the membrane be?
If we denote by $u(x,y)$ the displacement of it in the direction of the vertical axis ($z$) and $f$ the force per unit area applied (in the vertical direction), it can be seen that $u$ must satisfy the equation,
\beq 
\Delta u = f, \mbox{in } \Omega=\{(x,y)| \sqrt{x^2+y^2} < \mbox{drum radius}\}.
\label{ec2*}
\eeq

\ejer:
Convince yourself that~\ron{ec2*} is the equilibrium equation for the case where the applied force is small enough to produce a small deformation of the membrane.
Hint: First do the one-dimensional case.

Since the edge of the drum holds the membrane there, we will have the following {\bf boundary condition},
\beq 
u|_{\partial \Omega} =0, \;\; \partial \Omega=\{(x,y)\;|\; \sqrt{x^2+y^2} = 
\mbox{drum radius}\}.
\label{ec3*}
\eeq
Thus we have the following problem: Given $f$ in $\Omega$, find $u$ satisfying
\ron{ec2*} and \ron{ec3*}.
From our physical experience, we know that this problem should be solvable! Moreover, the problem should be solvable where the drum is not circular but has an arbitrary shape, allowing $\partial \Omega$ to be an arbitrary but smooth edge and also allowing $\partial \Omega$ not to be in the plane $z=0$, or -equivalently- to be in such a plane but allowing $u$ to take any --but smooth-- value $\phi$ on $\partial \Omega$.

We thus arrive at the following {\bf Dirichlet Problem:} 
Given $\Omega$ with smooth $\partial \Omega$, $f :\Omega \rightarrow \re$ smooth and
$\phi_0: \partial \Omega \rightarrow \re$, also smooth, find 
$u:\Omega \rightarrow \re$ satisfying,
\begin{enumerate}
\item $ \Delta u = f \;\;\;\;\mbox{in}\;\;\Omega$,
\item $u|_{\partial \Omega} = \phi_0.$
\end{enumerate}
Later we will reaffirm our intuition by seeing that this problem can always be solved.

Now suppose we allow the edges of the membrane to slide vertically but place vertical springs on the edge with a Hooke constant that depends on the position, $k:\partial \Omega \rightarrow \re$, and arranged in such a way that the equilibrium position before applying $f$ is $u=0$. When we apply $f$, the membrane will move until again in the interior we have,
\beq \Delta u = f \;\;\;\;\mbox{in}\;\;\Omega \eeq
and on the edge
\beq
(ku + n^a\nabla_a u)|_{\partial \Omega} = 0,
\eeq
where $n^a$ is the normal to $\partial \Omega$. 
This {\bf Mixed Problem} can also be solved. 
%\fig{6cm}{Mixed Problem.}
\begin{figure}[htbp]
  \begin{center}
    \resizebox{7cm}{!}{\myinput{Figure/m12_2}}
    \caption{Mixed problem.}
    \label{fig:12_2}
  \end{center}
\end{figure}

A particular case of this is when instead of using springs to apply the edge force, we simply use a given force $\phi_1 :\partial \Omega \rightarrow \re$, but taking care that its total contribution, $\int_{\partial \Omega} \phi_1 dS $, exactly cancels the total contribution of $f$ --otherwise we would have an accelerating membrane--. In such a case we have the {\bf Neumann Problem:}
\beqarr 
\Delta u &=& f \;\;\;\;\mbox{in}\;\;\Omega \\
n^a\nabla u|_{\partial \Omega} &=& \phi_1,\mbox{ on the edge} \;\; \partial \Omega,\\
\mbox{with} \dip\int_{\partial \Omega} \phi_1 \;dS 
&=& \dip\int_{\partial \Omega} n^a\nabla_a u \;dS
= \dip\int_{\Omega} \Delta u \;dV = \dip\int_{\Omega} f \;dV .
\eeqarr

\subsection{Existence}

Next, we will see that the Dirichlet problem always has a unique solution (assuming that $f$, $\phi_0$, and $\partial \Omega$ are sufficiently smooth). The other problems are solved similarly.

Suppose there exists $u \in H^2(\Omega)$ satisfying,
$\Delta u = f $ in $\Omega$ and $u|_{\partial \Omega} = 0$ --which implies 
$u \in H^1_0(\Omega)$--. Then using the divergence theorem we obtain the {\bf first Green's identity}, $\forall v \in C^{\infty}(\Omega). $

\beq \int_{\Omega} v\bar f \; d^nx = \int_{\Omega} v\Delta \bar{u} \;d^nx =
- \int_{\Omega} e^{ab} \nabla_a v \nabla_b \bar u \; d^nx +
\int_{\partial \Omega} v(n^a\nabla_a \bar u) \; d^{n-1}x, 
\eeq
%
If we assume that $v \in H^1_0(\Omega)$ the identity is still valid and in this case reduces to,
\beq
\int_{\Omega} e^{ab} \nabla_a v \nabla_b \bar u \; d^nx =
- \int_{\Omega} v \bar f \; d^nx , \;\;\;\;\;\forall \;\;v \in H^1_0(\Omega).
\label{ec4*}
\eeq
But note that this identity is still valid if we simply assume that $u \in H^1_0(\Omega)$ --not necessarily in $H^2(\Omega)$-- and that $f \in H^{-1}(\Omega)$. 

Thus we have the {\bf Weak Dirichlet Problem} 
(with $\phi_0 = 0$):
Find $u \in H_0^1(\Omega)$ such that given $f \in H^{-1}(\Omega)$, \ron{ec4*} is satisfied.

If the right-hand side of \ron{ec4*} were an inner product, this would give rise (by completion) to a Hilbert space $H$ and then 
 \ron{ec4*} would take the form 
\beq
 \langle u,v \rangle_H = \Phi_f(v) := -\int_{\Omega} \bar f v \; d^nx, \forall v
\in H^1_0(\Omega).
\label{ec5*}
\eeq
 
If $H = H^1_0(\Omega) $ and if $\Phi_f: H \rightarrow \ve C$ were
continuous, then the Riesz representation theorem would tell us that
there exists a unique $u$ in $H$ satisfying \ron{ec4*} and therefore \ron{ec5*}.
As we will see below (Poincaré-Hardy lemma), in this case, the right-hand side of \ron{ec4*} is an inner product equivalent to that of 
$H^1_0(\Omega)$
and therefore $H = H^1_0(\Omega) $. But then $\Phi_f$ is clearly
continuous since $H^{-1}(\Omega)$ is the dual of 
$H^1(\Omega) \supset H^1_0(\Omega)$
and $u \in H^1_0(\Omega)$. We have thus proved:

\bteo
[existence and uniqueness] Given $f \in H^{-1}(\Omega)$ there exists a
unique $u \in H^1_0(\Omega)$ satisfying the weak Dirichlet problem.  
\eteo
\bcor
  The map $(\Delta,\tau) : H^1(\Omega) \rightarrow H^{-1}(\Omega)
\times H^{1/2}(\Omega)$ given by,
\beq \barr{rcl}
\Delta u &=& f \in H^{-1}(\Omega),\\
\tau u &=& \phi_0 \in H^{1/2}(\pa \Omega),
\earr
\eeq 
is an isomorphism~\footnote{Always understanding these equations in
their {\sl weak} or distributional form}.
\ecor

\pru:
It is clear that the map is continuous --since it is linear and bounded--. 
It is also
injective since if $\phi_0 = 0$ then $u \in H^1_0(\Omega)$ and if $f=0$
the previous theorem (uniqueness) tells us that $u=0$.
Let's see that it is surjective. Let $\phi_0 \in H^{1/2}(\Omega)$, then there exists
$w \in H^1(\Omega) $ such that its restriction to $\pa \Omega$,  
$\tau w = \phi_0$ and let $f \in H^{-1}(\Omega)$. 
Since also $\Delta w \in H^{-1}(\Omega)$ the previous theorem
guarantees that there exists $\bar u \in H^1_0(\Omega)$ such that 
\beq
\Delta \bar u = f - \Delta w \mbox{ in } \Omega.
\eeq
But then $u = \bar u + w$ satisfies $\Delta u = f$ in $\Omega$ and 
$\tau u = \phi_0 $ on $\partial \Omega$.

\espa

\noi It only remains to prove then that the previous inner product is indeed an inner product and the norm thus obtained is equivalent to that of $H^{1}_{0}$. This follows from the following result,

\blem[of Poincaré-Hardy] 
\label{lem12.1}
There exists $C>0$ such that for all 
$u \in H^1_0(\Omega)$,
\beq
\int_{\Omega} |u|^2 \; d^nx \leq C \int_{\Omega} e^{ab} \nabla_a \bar u
\nabla_b u \; d^nx.
\eeq
\elem

This tells us that the inner product in $H^1_0(\Omega)$ is equivalent
to the inner product used previously.

\espa

\pru: Since $C^{\infty}_0(\Omega)$ is dense in $H^1_0(\Omega)$ it is sufficient
to prove the inequality for these functions. Let $\Gamma_d$
be an n-cube of side $d$ containing $\Omega$. Extending the functions
in $C^{\infty}_0(\Omega) $ as zero in $\Gamma_d - \Omega$ we obtain,
\beq\barr{rcl}
|u|^2(x) &=& |\int^{x^1}_{-d/2} \partial_1 u(\xi^1,x^2,...,x^n) \; d\xi^1|^2\\
         &\leq& (x^1+\frac d2)\int^{x^1}_{-d/2} |\partial_1u|^2 \; d\xi^1\\
         &\leq& d\int^{d/2}_{-d/2} |\partial_1 u|^2 \; d\xi^1.
\earr
\eeq
Therefore,
\beq 
\int^{d/2}_{-d/2}|u(x)|^2 dx^1 
\leq d^2 \int^{d/2}_{-d/2} |\partial_1 u|^2 \; d\xi^1,
\eeq
and
\beq
\int_{\Gamma_d} |u|^2(x) \; d^nx 
\leq d^2 \int_{\Gamma_d} |\partial_1 u|^2 \; d^nx 
\leq C \int_{\Gamma_d} \nabla_a \bar u \nabla^a u \; d^nx,
\eeq
with $C = d^2$. This proves the lemma and concludes the proof of existence
and uniqueness $\spadesuit\spadesuit$

Both the existence and uniqueness theorem and the regularity theorem
(which we give below)
can be generalized to elliptic equations with non-constant coefficients as long as they are smooth, it is satisfied that $c \leq 0$ and that
$a^{ab}l_al_b > k g^{ab}l_al_b$, where $k$ is a constant and
$g^{ab}$ is a positive definite metric such that the volume of
$\Omega$ with respect to this metric is finite. The proof we gave is
valid only if $Vol_g(\Omega) < \infty$, since we used the Poincaré-Hardy lemma.
If $\Omega = \re^n$ then the proof is still valid if we substitute
the space $H^1_0$ by the space $H^1_1$, that is, the space of functions that decay to infinity, with norm
$$
||f||^2_{H^1_1(R^n)} = \int_{R^n} \frac{|f|^2}{r^2} + e^{ab}\na_a
\bar{f} \na_b f.
$$

In this case, it can be proven that
the solution to the problem will not only be smooth but also decay asymptotically like $1/r$.

\subsection{*Regularity of Solutions}

The solution we have found is only in the weak sense. If we now assume that $f$ and $\phi_0$ have some regularity, we can also conclude that,

\bteo
[Regularity] Let $u \in H^1(\Omega)$ be a weak solution of the equation $\Delta u = f$ in $\Omega$ with boundary condition $u|_{\partial \Omega} = \phi_0$ and let $f \in H^k(\Omega)$, $\phi_0 \in H^{k+\frac32}(\partial \Omega)$, $k \geq -1$ then $u \in H^{k+2}(\Omega)$. In particular, if $f \in C^{\infty}(\Omega)$ and $\phi_0 \in C^{\infty}(\partial \Omega)$, then $u \in C^{\infty}(\Omega)$.
\eteo

Before proceeding with the proof, we will define the finite difference operator and see its properties. Let $\{x^i\}$ be a coordinate system in an open subset of $\Omega$.

\espa
\defi: 
\[
\Delta^h_i u(x^1,...,x^i,...,x^n) := \dip\frac{u(x^1,...,x^i+h,...x^n) - u(x^1,...,x^i,...,x^n)}{h}.
\]

\blem
\label{lem12.2}
 If $u \in H^1(\Omega)$ and $\Omega' \subset \subset \Omega$ (strictly contained, $\partial\Omega \cap \Omega' = \emptyset$) then,
\beq \|\Delta^h_i u \|_{H^0(\partial\Omega)} \leq \|\partial_i u \|_{H^0(\Omega)}.
\eeq
\elem

\espa

\pru: It is sufficient to consider the case $u \in C^1(\Omega)$. If $h<dist(\partial\Omega,\Omega')$, then 
\beq
\Delta^h_iu(x) = \frac1h \int_0^h \partial_i u(x^1,...,x^i+\xi,...,x^n) d\xi
\;\;\forall x \in \Omega',
\eeq
therefore, using the Schwarz inequality, we see that,
\beq 
|\Delta^h_iu(x)|^2
\leq \frac1h \int_0^h |\partial_i u(x^1,...,x^i+\xi,...,x^n)|^2 \; d\xi.
\eeq
Integrating over $\Omega'$ we obtain,
\beq
\int_{\Omega'} |\Delta^h_i u(x)|^2 d^nx \leq
\frac1h \int^h_0 \int_{\Omega'} |\partial_i u|^2 d^nx \; d\xi \leq
\int_{\Omega} |\partial_i u |^2 \; d^nx.
\eeq
$\spadesuit$

\blem
\label{lem12.3}
Let $u \in H_0(\Omega)$ 
(and therefore $\Delta^h_i u \in H_0(\Omega')$)
and suppose there exists $k<0$ such that 
$\|\Delta^h_i u \|_{H_0(\Omega')} \leq k \;\;\forall \;h>0$ and $\Omega'\subset
\subset \Omega$, with $h<dist(\partial\Omega,\Omega')$. 
Then the weak derivative $\partial_i u$ exists and satisfies,
$\|\partial_i u \|_{H_0(\Omega)} \leq k$.
\elem

\espa

\pru: By the weak compactness of closed and bounded sets in $H_0(\Omega')$, there will exist a sequence $\{h_m\}\rightarrow 0$ and $v_i \in H_0(\Omega)$ with $\|v_i\|_{H_0(\Omega')} \leq k$ such that 
$\Delta^{h_m}_i u \sr{d}{\rightarrow} v_i$, that is
\beq 
\int_{\Omega} \phi \Delta^{h_m}_i u \;d^nx \rightarrow 
\int_{\Omega} \phi v_i \;d^nx, \;\;\;\forall \;\phi \in C^1_0(\Omega).
\eeq
On the other hand, if $h_m <dist(sop.\phi ,\partial\Omega)$, then
\beq
\int_{\Omega} \phi \Delta^{h_m}_i u \; d^nx =
- \int_{\Omega} u (\Delta^{-h_m}_i \phi ) \; d^nx \rightarrow
- \int_{\Omega} u \partial_i \phi \; d^nx.
\eeq
We thus conclude that, 
\beq
\int_{\Omega} \phi v_i \; d^nx = - \int_{\Omega} u \partial_i \phi \; d^nx,
\eeq
meaning that $v_i$ is the weak (distributional) derivative of $u$ in the direction
$x^i$ $\spadesuit$

\espa

\pru: (Regularity Theorem):
We will be content to prove that $u$ is regular in any $\Omega'$
strictly contained in $\Omega$, the extension of the proof to
the whole $\Omega$ does not add any new concept. 
We will prove the statement for $k=0$, the rest follows by induction on $k$.
Since $u$ is a weak solution, we have that 
\beq 
\int_{\Omega} \nabla^a \bar u \nabla_a v \; d^nx = 
\int_{\Omega} \bar f v \; d^nx, \forall v \in C^1_0(\Omega).
\label{ec6*}
\eeq 
Replacing $v$ by $-\Delta^{-h}_i v$ with $|2\,h|<dist(sop.v ,\partial\Omega)$,
we have,
\beq 
\int_{\Omega} (\Delta^h_i\nabla^a \bar u) \nabla_a v \; d^nx = 
- \int_{\Omega} \bar f \Delta^h_i v \; d^nx, \forall v \in C^1_0(\Omega)
\eeq
and therefore (Using Lemma~\ron{lem12.2}),
\beq 
\|\int_{\Omega} (\Delta^h_i\nabla^a \bar u) \nabla_a v \; d^nx\| \leq 
\|f\|_{H_0} \|\partial_i v\|_{H_0}.
\eeq
Now taking $v = \Delta^h_i u $ we obtain,
\beq 
\|\int_{\Omega} (\Delta^h_i\nabla^a \bar u) \Delta^h_i \nabla_a u \; d^nx\| \leq 
\|f\|_{H_0} \|\partial_i \Delta^h_i u\|_{H_0},
\eeq
that is,
\beq 
\|\Delta^h_i\nabla^a u \|^2_{H_0}  \leq 
\|f\|_{H_0} \|\Delta^h_i \:\na_a u\|_{H_0},
\eeq
which finally implies,
\beq 
\|\Delta^h_i\nabla^a u \|_{H_0}  \leq \|f\|_{H_0} < k. 
\eeq
Lemma~\ron{lem12.3} then tells us that 
$\partial_i u \in H^1(\partial\Omega)$ and
completes the first part of the proof.

Now, to complete the proof, let's see that if $f \in C^{\infty}(\Omega)$ then $u \in C^{\infty}(\Omega)$.
But this is obvious since if $u \in H^p(\Omega)$, $\Omega \subset \re^n$, 
then by Sobolev's Lemma $u \in C^{p - \frac n2 - \eps} (\Omega)
\;\;$ $\forall \;\eps > 0$ and in particular if $u \in H^p(\Omega) \;\;\;\;
\forall \;p$ then
$u \in C^{\infty}(\Omega)$ $\spadesuit$
\espa




\section{Spectral Theorem}
\label{sec:Spectral-Theorem}

Consider a rod made of a material with thermal conductivity $q=1$ and length
$L=2\pi$ and suppose we want to describe the evolution of its
temperature distribution $T(t,x)$. To do this, we will assume an
initial distribution $T_0(x)$ and that the ends of the rod are
connected to an infinite heat reservoir at zero degrees. We must
then solve the mathematical problem,
\beq\barr{rcl}
\dip\frac d{dt} T &=& - \dip\frac {d^2}{dx^2} T, \;\;\;t\geq 0 \\ [3mm]
T(t_0,x) &=& T_0(x),\\ [3mm]
T(t,0) &=& T(t,2\pi) = 0\;\;\;t \geq 0.
\earr\eeq
To solve it, we will use a Fourier series expansion, that is,
we will propose a solution of the form,
\beq
T(t,x) = \sum^{\infty}_{n=1} C_n(t) \sin(\frac{nx}2),
\eeq
which clearly satisfies the boundary conditions.
Applying the equation and using the orthogonality of the functions 
we obtain,
\beq
\frac d{dt} C_n(t) = - \frac{n^2}4 C_n(t),
\eeq
which has the solution,
\beq
C_n(t) = C_n(0)e^{-\frac {n^2}4 t}.
\eeq 
Therefore, if we give the initial condition 
$T_0 \in L^2(0,2\pi)$, it will determine a sequence 
$\{C^0_n = (T_0,\sin(\frac{nx}2))_{L^2}\}$ in $l^2$ that
we will use as the initial condition and thus obtain $T(t,x)$.
Since 
\beq
\sum^{\infty}_{n=1} |C_n(t)|^2 = \sum^{\infty}_{n=1}
|C_n^0|^2e^{-\frac{n^2}2 t} \leq \sum^{\infty}_{n=1} |C_n^0|^2 < \infty
\eeq
the formal solution --since we do not know if it is differentiable-- will be
in $L^2(0,2\pi)$ for all $t \geq 0$. For any $t>0$ and $q \in \mathbb{N}$ 
we have that $n^{2q}e^{-\frac{n^2}2 t}$ tends
exponentially to zero as $n$ tends to infinity and therefore
$T(t,x) \in H^q(0,2\pi)$ which implies $T(t,x) \in C^{\infty}((0,+\infty)\times
[0,2\pi])$~\footnote{It can also be shown that $T(t,x)$
is analytic in both variables in $(0,+\infty)\times
[0,2\pi])$}.

\noi We have thus completely solved this problem.

\espa
What have we used to construct these solutions?
We have used that any function that vanishes at the ends can
be expanded in terms of its Fourier series, that is, in terms of the
functions $\sin(\frac{nx}2)$ and also that,
\beq
\frac{d^2}{dx^2} \sin(\frac{nx}2) = -\frac{n^2}4 \sin(\frac{nx}2)
\eeq
In analogy with the theory of operators between finite-dimensional vector spaces,
we will call the above equation the eigenvector-eigenvalue equation of the linear operator $\frac{d^2}{dx^2}$. 
Given a
linear operator $L:D \subset L^2 \rightarrow L^2$ 
the problem of finding its eigenvalues and
eigenvectors with given boundary conditions is called the Sturm-Liouville problem. As we will see below, there is a large
number of operators for which this problem has a solution.
Is it a coincidence that the set of eigenvalues of the operator in the
example is a basis of $L^2$?~\footnote {To obtain pointwise convergence
(and not just in $L^2$) in the case of non-zero boundary conditions, 
one must add the eigenvectors (with zero eigenvalue) 
$f_0(x) = 1$ and $f_1(x)=x$.} 
The following theorem and its corollary tell us that it is not, if the
operator in question satisfies certain conditions.

\bteo[Spectral for the Laplacian]
Let $\Omega \subset \mathbb{R}^n$ be bounded. 
Then the Laplacian has a countable and discrete set of eigenvalues,
$\Sigma=\{\lam_i\}$,
whose eigenfunctions (eigenvectors) expand $H^1_0(\Omega)$.
\eteo

% To prove this theorem we need the following lemmas:


\pru: [Spectral Theorem]
Consider the functional
${\cJ}:H^1_0(\Omega) \rightarrow \mathbb{R}^+$ given by,
\beq
\mbox{\cJ}(u) = \frac{\int_{\Omega} \na_a \bar u \na^a u \;d^nx}
{\int_{\Omega} \bar u u \:d^nx}.
\eeq
Since $\cJ (u)$ is non-negative, there will exist $\lam_0(\Omega) \in \mathbb{R}^+$ such that 
\beq 
\lam_0 = \barr{c} \\ inf \cJ (u) \\^{
       u\in H^1_0(\Omega)} \earr.
\eeq

\ejer: 
Relate this $\lam_0$ with the constant in the
Poincaré-Hardy Lemma. 
\espa

% filepath: /Users/reula/Docencia/Metodos_libro/libro_metodos_github/translated_text.tex
As we will see below, this $\lam_0$ is the smallest of the eigenvalues of the Laplacian, and its eigenfunction $u_0$ is the one that minimizes \cJ. 
Indeed, suppose there exists $u_0$ in $H^1_0(\Omega)$ that minimizes \cJ. 
Since \cJ is a differentiable functional, we must have
\beq
\frac d{ds} \cJ (u_0 + sv)|_{s=0} = 0 \;\;\;\; \forall\;v 
\in H^1_0(\Omega).
\label{des}
\eeq
But 
\beqarr
\frac d{ds} \cJ (u_0 + sv)|_{s=0} 
&=& \frac{1}{\int_{\Omega} \bar u u \:d^nx} 
[\int_{\Omega}(\na^a \bar v \na_a u_0 + \na^a \bar u_0 \na_a v) d^nx \nonumber \\
&-&
\lambda_0 [\int_{\Omega}(\bar v u_0 + \bar u_0 v) d^nx]]
\eeqarr
and therefore \ron{des} is equivalent to $u_0$ satisfying the weak version of the eigenvalue-eigenvector equation. 
Now let's see that $u_0$ exists. 
Since $\lam_0$ is the infimum of \cJ, there exists a sequence 
$\{u_p\} \in H^1_0(\Omega)$
such that $\|u_p\|_{H^0(\Omega)} = 1$ and $\cJ (u_p) \rightarrow \lam_0$.
But such a sequence is bounded in $H^1_0(\Omega)$ and therefore by 
lemma~\ron{lem10.10} there exists a subsequence $\{\ti u_p\}$ that converges 
strongly in $H^0(\Omega)$
to a function $u_0$ with $\|u_0\|_{H^0(\Omega)} = 1$. 
Since $\mbox{\cal Q}(u) := \int_{\Omega} \na^a \bar u \na_a u \; d^nx$ is a norm 
derived from an inner product,
the parallelogram law holds,
\beq
\mbox{\cal Q}(\frac{\ti u_p - \ti u_q}2) + \mbox{\cal Q}(\frac{\ti u_p + \ti u_q}2) =
\frac12(\mbox{\cal Q}(\ti u_p) + \mbox{\cal Q}(\ti u_q)),
\eeq
which implies 
\beq
\mbox{\cal Q}(\frac{\ti u_p - \ti u_q}2) \leq
\frac 12 (\mbox{\cal Q}(\ti u_p) + \mbox{\cal Q}
(\ti u_q)) - \lam_0 \|\frac{\ti u_p +\ti u_q}2\|^2_{
H^0(\Omega)} 
\barr{c} \\ \rightarrow \lam_0 - \lam_0 = 0.\\^{
p,q \rightarrow \infty}
\earr
\eeq
Where we have used that since $\{\ti u_{q}\} \to u_{0}$ in $H^{0}(\Omega)$, $\{\frac{\ti u_p +\ti u_q}2\}$ also does,
and therefore $\{\|\frac{\ti u_p +\ti u_q}2\|_{H^0(\Omega)}\} \to 1$.
But since the norm $\mbox{\cal Q}(u)$ is equivalent to that of $H^1_0(\Omega)$, 
we see that $\{\ti u_p\}$ is 
also Cauchy in $H^1_0(\Omega)$ and therefore $u_0 \in H^1_0(\Omega)$. 
Using the regularity theorem with $f= \lam_0u_0$, we see that $u_0 \in C^{\infty}(\Omega) \cap 
H^1_0(\Omega)$ and therefore $u_0$ is an eigenvalue
in the classical sense $(\Delta u_0 + \lam_0 u_0 = 0)$. 
Now let's prove the existence of the other
eigenvalues-eigenvectors.
Let $H(1) = \{ u \in H^1_0(\Omega) | \langle u , u_0 \rangle_{H^0(\Omega)} = 0\}$.
This is a vector subspace and is closed, therefore it is a
Hilbert space~\footnote {Note that $H(1) \neq $ the perpendicular space to 
$u_0$ in the $H^1_0(\Omega)$ norm.}.
This space is invariant with respect to the action of the Laplacian. 
Indeed, if $v \in H(1)$, then,

\begin{eqnarray*}
\langle u_0 , \Delta v \rangle &=& \int_{\Omega} \bar{u}_0 \Delta v \;d^n x \\
                &=& \int_{\Omega} \Delta \bar{u}_0 v \;d^n x + \int_{\partial \Omega} [\bar{u}_0 n^a \nabla_a v + v n^a \nabla_a \bar{u}_0] \; d^{n-1} x \\
                &=&  \int_{\Omega} \Delta \bar{u}_0 v \;d^n x \\
                &=& \lambda_0 \int_{\Omega} \bar{u}_0 v \;d^n x \\
                &=& \lambda_0 \langle u_0, v \rangle, \\
                &=& 0
\end{eqnarray*}
%
which tells us that $\Delta v \in H(1)$.                
Repeating the previous proof, we conclude that 
therefore there will exist $\lam_1$, such that
\beq 
\lam_1 = \barr{c} \\ inf \\^{u \in H(1)} \earr
              \cJ (u).       
\eeq
%
and that $\lam_1$, will be an eigenvalue, that is, there will exist an eigenfunction $u_1$, with eigenvalue $\lam_1$. 

Defining $H(2) = \{u \in H^1_0(\Omega) | \langle u , u_0 \rangle_{H^0(\Omega)}= \langle u , u_1 \rangle_{H^0(\Omega)}=0
\}$, etc. 
we can continue indefinitely and obtain $\Sigma$ and, correspondingly, an orthonormal set \footnote {After normalizing them appropriately.}, in
$H^0(\Omega)$, of eigenvectors.
To complete the proof, let's see that this set expands $H^1_0(\Omega)$.
Since $u_i$ satisfies 
$\Delta u_i +\lam_i u_i =0 $ and $\langle u_i , u_j \rangle_{H^0(\Omega)} = 0 $ if $i \neq j$ 
we have that
\beq
0= \langle u_j , \Delta u_i + \lam_i u_i \rangle_{H^0(\Omega)} = 
- \int_{\Omega} \na^a \bar u_j \na_a u_i d^nx
\eeq
and therefore also that
\beq
\langle u_j , u_i \rangle_{H^1_0(\Omega)} = 0.
\eeq
We then see that this set is orthogonal in $H^1_0(\Omega)$ and that by
construction its perpendicular subspace is $\{ 0 \}$, which implies that
this set is a basis of $H^1_0(\Omega)$ $\spadesuit$


\espa
\ejer: 
Find the eigenvalues and eigenvectors of the Laplacian in
$H^1_0(\Omega)$ when:
a) $\Omega \subset \re^2$ is a square with side $L$.
b) $\Omega \subset \re^3$ is a sphere with radius $R$.
Construct in both cases using these eigenfunctions the Green's function of
the problem in question.

For which equations can the previous theorem be generalized? For
the proof, specific properties of the Laplacian were used only to
assert that $\cJ$ was bounded below --to conclude that the infimum
existed-- and that $\mbox{\cal Q}(u)$ was a norm derived from an inner product
--to conclude that the parallelogram law held--.
If 
\beq
L(u)= a^{ab} \na_a \na_b u + b^a \na_b u + cu,                 \label{ec*}
\eeq 
%
with $a^{ab}$, $b^a$ and $c$ smooth (real) fields in $\Omega$, and such that there exists $k>0$ such 
that
$a^{ab} l_al_b \geq \;k\; g^{ab} l_al_b$, for every vector field 
$l_a$ in $\Omega$ (ellipticity)
then there exist positive constants $c_1$ and $c_2$ such that 
\beq
\langle u , -L(u) \rangle_{H^0(\Omega)} = -\int_{\Omega} \bar u L(u) d^nx 
                     \leq c_1 \int_{\Omega} g^{ab}\na_a\bar u \na_b u d^nx
                     -c_2 \int_{\Omega} |u|^2 d^nx.
\eeq                 
\ejer: Prove this.
\espa

\noi
Therefore in this case we also have that 
\beq
\cJ (u) := \frac{\langle u , -L(u) \rangle_{H^0(\Omega)}}{\langle u , u \rangle_{H^0(\Omega)}},
\eeq
is bounded below.
The condition that $\mbox{\cal Q}(u) := \langle u , -L(u) \rangle_{H^0(\Omega)}$ 
satisfies the \textsl{parallelogram law}, even if
it is not positive definite, is much more restrictive, and is equivalent to
requiring that $L$ satisfies
\beq
\langle v , L(u) \rangle_{H^0(\Omega)} = \langle L(v) , u \rangle_{H^0(\Omega)}\;\;\;\forall\;u,v \in H^1_0(\Omega).
\eeq
Operators that satisfy this relation are called {\bf
self-adjoint} or {\bf Hermitian}.
Note that this condition also ensures that the operator $L$ leaves invariant the respective subspaces $H(i)$ that need to be 
considered in the previous proof.

\ejer: 
Show that
\beq
L(u)= \na_a(a^{ab} \na_b u + b^a  u) - b^a \na_a u + cu,
\eeq 
with $a^{ab}$, $b^a$ and $c$ real tensor fields is self-adjoint.

\ejer:
Show that if $L$ is self-adjoint then the eigenvectors
are real.

\ejer: 
Find the eigenvalues and eigenvectors in $H^1_0(\Omega)$ 
of 
\beq
L(u) = \frac{d^2}{dx^2}u + c x^2 u.
\eeq

\noi
We thus arrive at the following generalization:

\bteo[Spectral] 
Let $\Omega$ be bounded and $L$ an elliptic
self-adjoint operator with coefficients $a^{ab}$, $b^a$ and $c$ in $C^{\infty}(\Omega)$
[A condition that can be considerably weakened]. 
Then the eigenvalue problem 
$L(u_i) = \lam_i u_i$, $u_i \in H^1_0(\Omega)$,
has a countable and discrete set of real eigenvalues, whose
eigenfunctions $u_i \in C^{\infty}(\Omega)$ expand $H^1_0(\Omega)$.
\eteo

\ejer:

\noi 
a) Prove the following {\bf corollary:}

If $L$ is elliptic and self-adjoint such that its eigenvalues are
different from zero, then the Dirichlet problem
\beq
L(u) = f,
\eeq
$f\in H^0(\Omega)$, $u \in H^1(\Omega)$,
has a unique solution.

\noi 
b) If some $\lam_i = 0$ then the previous problem has a solution 
iff $\langle u_i , f \rangle_{H^0(\Omega)} = 0$ 
for every eigenfunction with zero eigenvalue.

%%% Local Variables: 
%%% mode: latex
%%% TeX-master: "apu_tot.tex~/Metodos/"
%%% End: 
