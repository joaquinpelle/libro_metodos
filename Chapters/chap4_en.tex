%\imput format
\chapter{Ordinary Differential Equations}
\label{Ecuaciones_Diferenciales_Ordinarias}




\section{Introduction}

In this chapter we will begin the study of systems of 
ordinary differential equations --ODE from now on--.
These systems constantly appear in physics and together with
the systems of partial differential equations, $\;\;$--PDE
from now on-- that we will study in the second part of this course,
form the mathematical skeleton of the exact sciences. This is because
most physical phenomena can be
described in terms of these equations. 
If the free parameter --or independent variable-- has the
interpretation of being a time, then we are dealing with
{\bf evolution equations}. 
These allow us, for each initial state of the system, 
to know its temporal evolution, that is, they give us the ability to
make predictions from initial conditions, which is the
characteristic aspect of any physical theory.~\footnote{
In some cases, the independent variable is not time
but some other parameter, but the equation can also be 
considered as describing evolution. For
example, if we want to see how the density of a medium varies when
the temperature increases.}

There are other physical phenomena described by ODE that do not have
an evolutionary character. In general, in these cases, we are
interested in equilibrium states of the system. 
For example, if we want to see what shape the clothesline at home describes.
These cases will not be included in the
theory of ODE that we will develop below, but 
their simplest cases will be
treated as special (one-dimensional) cases of elliptic PDE.


Although quantum physics, or more precisely quantum field theory,
shows us that a description of physical phenomena
at that microscopic scale is not possible through the systems of
differential equations mentioned above, we usually manage
to find partial aspects of these phenomena, or certain
approximations where a description in terms
of these systems is possible. This is not due to a whim or stubbornness, 
but to the fact that
our knowledge of ODE, and to a lesser extent of PDE, is
quite deep, which allows us to handle them with relative ease
and comfort.
On the other hand, and it is important to emphasize this, our knowledge of the
type of equations that appear in non-linear quantum field theories
is, in comparison, practically nil.



\section{The Case of a Single First-Order Ordinary Equation}

\subsection{First-Order Autonomous Equation}


This has the form:
\beq
\dot x= \frac{dx}{dt}=f(x)   \label{eq:4.5}
\eeq
\noindent 
that is, an equation for a map $x(t):I \subset \re
\to U\subset \re$.

\noindent We will assume $f:U\to \re$ continuous. A solution of this
equation will be a differentiable map $\varphi (t)$ defined for 
all $I$ and with image in $U$.
We will say that $\varphi (t)$ satisfies the {\bf initial condition}
$x_0$  at $t=t_0 \in I$ if 
\beq 
\varphi (t_0)=x_0 .   \label{eq:4.6}
\eeq   
As we will see below, under certain hypotheses, the initial
condition that a solution satisfies determines it uniquely.



The equation~\ref{eq:4.5} can be interpreted geometrically in the
following way. At each point of the plane $(x,t)$ we mark a
"little line", that is, a field of directions such that the
angle it forms with the axis $x=$cte has a tangent equal to $f(x)$.

\ejem: $\dot x = x^{\frac{1}{2}}$. [See figure (\ref{fig:4_1}).]


%\fig{5cm}{Interpretación geométrica de la ecuación  $f(x)=x^{1/2}$.}
\begin{figure}[htbp]
  \begin{center}
    \resizebox{7cm}{!}{\myinput{Figure/m4_1}}
    \caption{Geometric interpretation of the equation  $f(x)=x^{1/2}$.}
    \label{fig:4_1}
  \end{center}
\end{figure}

Since $\varphi (t)$ has a derivative $f(\varphi(t))$, at the point
$(\varphi(t),t)$, $\varphi(t)$ will be tangent to the little line at that
point. If we stand at some point $(x_0,t_0)$ through which it
passes, we can find the solution by following the little lines. Therefore,
we see that given a point through which the solution passes, by following
a little line, we will determine a unique solution. This, in particular,
tells us that in this graph, most solutions do not intersect.
As we will see later, the uniqueness of the solution occurs if the little line
at each point has a non-zero tangent. 
On the other hand, it is clear from the case in the previous figure, \ref{fig:4_1}, that if
we start with the point $(x_0=0,t_0)$, we would have two options: either
draw the already drawn line $(t^2,t)$ or simply the line $(0,t)$. 
\espa

\ejer: 
Extend this graphical intuition to the case $\dot{x} = f(x,t)$.

\ejem:
In fact, the equation $\dot x= x^{1/2}$ has two solutions that pass
through $(0,0)$, 
\beq 
\varphi(t)=0 \,\,\mbox{y}\,\,\,\varphi=\frac{t^2}4 
\eeq 


Indeed, suppose $x(t)\neq 0$ then $\frac{dx}{x^{1/2}}=dt$ or
$2(x^{1/2}-x_0^{1/2})=t-t_0$, taking $x_0=t_0=0$ 
\beq 
x=\left(\frac t2 \right)^2
\eeq 
The other is trivially a solution.
\espa

\par
The uniqueness of the solutions is obtained, even in the case where $f(x)$ 
vanishes, if instead of asking that $f(x)$
be continuous, we ask a little more, and that is that, where it vanishes,
say at $x_0$, it is fulfilled that 
\beq 
\lim_{\epsilon\to 0} \int_{x_0 + \eps}^x \frac {dx}{f(x)}
=\infty , 
\eeq 
or alternatively that for $\epsilon > 0$ sufficiently small,
there exists $k>0$ such that,
\beq 
|f(x)| <k\mid x-x_0 \mid  \,\,\,\mbox{si}\, \mid
x-x_0\mid<\epsilon.
\eeq  

That is, $f(x)$ goes to zero when $x\to x_0$ at most as
quickly as a linear function. This condition is called
the Lipschitz condition. From now on, we will assume that $f(x)$ is
differentiable and therefore Lipschitz.

\ejer: See that this condition is
weaker than asking that $f$ be differentiable at $x_0$.

 
\noindent {\underbar {Solution}}: The function $f(x)=x\sin (\frac
1x)$ is Lipschitz at $x_{+0}$ but not differentiable.


Thus, we arrive at our first theorem on ordinary equations.



\begin{teo} 
Let $f(x):U\to\re$ be continuous and Lipschitz at the points
where it vanishes. Then


i) For each $t_0\in\re$, $x_0\in U$ there exists an interval
$I_{t_0}\in\re$ and a solution $\varphi (t)$ of the
equation~\ref{eq:4.5} defined $\forall\,\,t\in I_{t_0}$ and the
initial condition $\varphi (t_0)=x_0$ ;


ii) Any two solutions $\varphi_1,\varphi_2 $ of~\ref{eq:4.5}
satisfying the same initial condition coincide in some
neighborhood of $t=t_0$


iii) the solution $\varphi $ of~\ref{eq:4.5} is such that
\beq 
t-t_0=\int_{x_0}^{\varphi(t)} \frac{dx}{f(x)}\:\:\:\mbox{si}\,\,
f(x)\neq 0
\eeq 
\label{teo:2}
\end{teo}

\noindent{\underbar {Note:}} The solution whose existence is guaranteed
by this theorem is valid in an open interval $I_{t_0}$, which the
theorem does not determine. Therefore, uniqueness occurs only in the
maximum interval that they have in common.








\pru:

If $f(x_0)=0$ then $\fit=x_0$ is a solution.

If $f(x_0)\neq 0$ then by continuity there exists a neighborhood of $x_0$
where $f(x)\neq 0$ and therefore $1/f(x)$ is a continuous function,
this implies that the integral in~\ref{eq:4.3} is a differentiable function
of its integration limits, which we will call
$\psi(x)-\psi(x_0)$.

\noi Therefore, we have,
\beq 
t-t_0=\psi(x)-\psi(x_0)
\eeq 

\noi That is, given $x$ we obtain $t$ and also this relationship is automatically fulfilled when
{$x=x_0,\,t=t_0$}. We now want to solve for $x$ from this relationship, that is,
find $\phi(t)$. 


\noi But
\beq 
\left. \frac{d\psi}{dx}\right|_{x=\zeta}=\frac1{f(\zeta)} \neq 0\,\,,
\eeq 
\noi Therefore, the inverse function theorem assures us that
there exists an $I_{t_0}$ and a \emph{unique} $\fit$ differentiable, defined 
for all $t\in I_{t_0}$ and such that  $\fip(t_0)=x_0$ and
$t-t_0=\psi(\fit)-\psi(x_0)$. 
Taking its derivative, we obtain,

\beq 
\frac{d\fip}{dt}=\left.\frac{d\psi^{-1}}{dt}\right|_{x=\fit}= 
\left.\left[\frac1{f(x)}\right]^{-1}\right|_{x=\fit} =f(\fit)
\eeq 
\noi which shows that it is a solution, necessarily unique
of~\ref{eq:4.5} with the appropriate initial condition.


It only remains to see that in the case where $f(x_0)=0$ the solution is unique.
For this, we will use the following Lemma.

\begin{lem} 
Let $f_1 $ and $f_2$ be real functions in
$U\su \re$ such that $f_1(x)<f_2(x)\;\;\forall\,x\in U$ and let $\fip_1$
and $\fip_2$ be the respective solutions to the differential equations that
they generate, defined in an interval $I\in \re$ and such that
$\fip_1(t_1)= \fip_2(t_1)$ for some $t_1\in I$. Then
\beq
\fip_1(t)\leq \fip_2(t)\;\;\;\;\;\forall\; \;t>t_1\,\in I     \label{eq:4.7}
\eeq
\label{lem}
\end{lem}

\espa


\pru: Trivial, the one who runs faster in each sector of the path wins!
Let the set $S = \{t>t_1 \in I | \phi_1(t) > \phi_2(t)\}$. 
Let $T \geq t_1$, $T \in I$ be the greatest lower bound of $S$.
 At this moment
$\fip_1(T)=\fip_2(T)$, 
but 
\beq 
\left.\frac{d\fip_1}{dt}\right|_{t=T}<\left.\frac{d\fip_2}{dt}\right|_{t=T}
\eeq 
%
Therefore, there exists $\eps > 0$ such that for all $t \in [T,\eps)$, 
$\fip_1(t)<\fip_2(t)$ which is
a contradiction, unless it is the last value in $I$.

%\fig{5cm}{Prueba del Lema 4.1.}
\begin{figure}[htbp]
  \begin{center}
    \resizebox{7cm}{!}{\myinput{Figure/m4_2_b}}
    \caption{Proof of Lemma \ref{lem}.}
    \label{fig:4_2}
  \end{center}
\end{figure}

With the help of this Lemma, we will prove the uniqueness of the solution.
Let $\fip_1(t)$ be a solution with $\fip_1(t_0)=x_0$ but different from
the solution $\fip(t)=x_0\;\forall\;t$. 
Suppose then that there exists
$t_1>t_0$ such that $\fip_1(t_1)=x_1\,>\,x_0$, the other cases are
treated similarly. [See figure 4.3.] 
%\fig{5cm}{Prueba de la unicidad de la solución.}
\begin{figure}[htbp]
  \begin{center}
    \resizebox{7cm}{!}{\myinput{Figure/m4_2}}
    \caption{Proof of the uniqueness of the solution.}
    \label{fig:4_2_b}
  \end{center}
\end{figure}

If we choose $x_1$ close enough to $x_0$, it is fulfilled that
\beq 
f(x)<k\mid x-x_0\mid\;\;\forall \;\; x_0\leq x\leq x_1
\eeq 
and therefore that $\fip_2(t) < \fip_1(t),\,\forall\;t_0\leq
t\leq t_1$, where $\fip_2(t)$ is the solution of the equation $\dot
x= k(x-x_0)$ with the initial condition $\fip_2(t_1)=x_1$. 
But $\fip_2(t)=x_0+(x_1-x_0)\, \mbox{e}^{k(t-t_1)}$ which tends to $x_0$
only when $t\to -\infty$. Therefore, $\fip_1(t)$ cannot take the value
$x_0$ for any finite $t < t_1$ and therefore we have a
contradiction \epru

The solution whose existence and uniqueness we have just demonstrated can
not only be thought of as a map between $I_{t_0}$ and $U$ but also
as a map $g_x^t$ between $U_{x_0}\times\,I_{t_0}$ and $U$ 
($U_{x_0}$ neighborhood of $x_0$). The map $g_x^t$ takes $(x,t)$ to the point
$\fit$ where $\fip$ is the solution of~\ref{eq:4.5} with the initial
condition $\fip (0)=x$.

These maps, or local diffeomorphisms, are called 
{\bf local phase flows}, 
(since in general they cannot be
defined in all $I_{t_0}\times U$) and satisfy the following
important group property; if $t,s \in I_0$ then
$g^{s+t}_{\,\,x}=g^s o g^t_{\,x}$.


By the way the initial conditions appear in the previous Theorem,
it is easy to see, using again the inverse function theorem, 
that if $f(x_0)\neq 0$ then $g^t_{\,x}$ is also differentiable 
with respect to $x$.

\begin{cor} if $f(x_0)\neq 0$ then
$g^t_{\,x}:U_{x_0}\times I_{t_0}\to U$ is differentiable in both $t$ and
$x$.
\label{cor1}
\end{cor}
\noi {\underbar {Note:}} The restriction $f(x_0)\neq 0$ is
unnecessary as we will see later.
\espa

From the use of the inverse function theorem in Theorem~\ref{teo:2}, it also
follows that if we consider the set of solutions distinguishing
each one by its initial condition,
[$\fip(x_0,t_0,t)$ is the solution that satisfies $\fip(t_0)=x_0$],
then the function $\fip(x_0,t_0,t):U\times I\times I\to U$ is
differentiable with respect to all arguments. That is, if
we slightly change the initial conditions, then the solution
will only change slightly. Later we will give a proof of all
the properties, including Theorem~\ref{teo:2} for general systems,
that is, equation~\ref{eq:4.4}.
\espa


\ejem:
Let $x(t)$ be the number of bacteria in a culture vessel at time $t$. If the number is large enough, we can think of it as a smooth, differentiable function. If this number is not so large that its density in the vessel is low enough so that they do not have to compete for air and food, then their reproduction is such that $x(t)$ satisfies the equation,
\beq
\dot x = a\,x\;\;\;\;\;\;a=\mbox{const.}\simeq 2.3\frac1{\mbox{day}},
\label{eq:4.8}
\eeq
that is, the growth rate is proportional to the number of bacteria present. The constant $a$ is the difference between the reproduction rate and the extinction rate. Applying Theorem~\ref{teo:2}, we know that the equation allows us, given the number of bacteria $x_0$ at time $t_0$, to know the number of bacteria at any time. Indeed, the solution of~\ref{eq:4.8} is 
\beq 
\frac{dx}x = a\,\,dt
\eeq 

\noi Integrating both sides
\beq 
\ln \frac x{x_0}=a\,(t-t_0) \;\;\;\mbox{or}\\;\;\;
x(t)=x_0\,\mbox{e}^{a\,(t-t_0)} 
\eeq 

That is, the number of bacteria grows exponentially, unless of course $x_0=0$, and depends continuously on the initial number. It should be noted that although the dependence is continuous, the difference between solutions with different initial conditions grows exponentially.

The phase flow in this case is $g^t_x=x\,\mbox{e}^{a\,t}$ and here it is global. With this growth, it is easy to realize that in a few days the number of bacteria and therefore their density will be so large that they will start to compete with each other for food, which will decrease the growth rate. It is experimentally observed that this fact can be taken into account by including in the~\ref{eq:4.8} a term proportional to the square of the number of bacteria.

\beq
\dot x= a\,x-b\,x^2\;\;\;\;\;\;\;\;\;b=\frac a{375}.
\label{eq:4.9}
\eeq

If $x$ is small, the first term dominates and $x$ begins to grow, as we have seen, exponentially until the second term becomes important and the growth rate decreases, tending to zero when $x=x_s$ such that $a\,x_s-b\,x_s^2=0$ or $a-b\,x_s=0$ or $x_s=a/b\simeq 375$ bacteria. Since $\fit=x_s$ is a solution, due to uniqueness, the solution that tends to $x_s$ that we described above cannot reach the value $x_s$ in finite time. If the number of bacteria present is greater than $375$, the growth rate will be negative and the solution will again tend asymptotically to $x_s$.

The general solution of~\ref{eq:4.9} can be obtained similarly to that of~\ref{eq:4.8} and is,
\beq
x(t)=\frac{a/b}{1+(\frac a{bx_0}-1)\mbox{e}^{-a(t-t_0)}}
\eeq 
\noi which confirms the previous analysis.

\espa

%\fig{6cm}{Different solutions of the bacterial growth equation: $\fip_1(t_0)=0$, $\fip_2(t_0)=x_s=\frac ab$, $\fip_3(t)=x_0<x_s$ and $\fip_4(t)=x_0>x_s$}

\begin{figure}[htbp]
  \begin{center}
    \resizebox{7cm}{!}{\myinput{Figure/m4_3}}
    \caption[Different solutions of the bacterial growth equation]{Different solutions of the bacterial growth equation:
$\fip_1(t_0)=0$, $\fip_2(t_0)=x_s=\frac ab$, $\fip_3(t)=x_0<x_s$ and
$\fip_4(t)=x_0>x_s$}
    \label{fig:4_3}
  \end{center}
\end{figure}

This example allows us to introduce two key concepts in the area.
The first is that of a {\bf stationary or equilibrium solution}. These are the constant solutions, that is, they are such that
$\dot x_s=f(x_s)=0$, that is, the roots of the function $f(x)$.
For equation~\ref{eq:4.8} $x_s=0$, for equation~\ref{eq:4.9}
$x_s=0\;\;\mbox{and}\;a/b$. 

Note that stationary solutions do not always exist, for example, the equation with $f(x)=1$ does not have them, but when they do exist, the problem of finding them reduces to solving at most a transcendental equation.
Due to the uniqueness of the solutions, this allows us to divide the
plane $(x,t)$ into strips such that if a solution has initial conditions in this strip, it must remain in it.

The second concept is the {\bf stability} of these solutions.
The solution $x_s=0$ of~\ref{eq:4.8} and~\ref{eq:4.9} is not stable in the sense that if we choose as an initial condition a point in any neighborhood of this solution, the corresponding solution will initially move away exponentially from the previous one. On the contrary, the solution of~\ref{eq:4.9} $x_s=a/b$ is stable in the sense that if we take initial values close to it, the corresponding solutions will asymptotically approach (in the future) the previous one.

Unstable solutions have no physical interest since the slightest perturbation of the system will generate a solution that quickly moves away from the unstable solution. 
This is not the case in the previous example since the number of bacteria is discrete and the equation only represents an approximation. 
If the culture vessel is sterilized and hermetically sealed, then there is no possibility of bacterial growth.
Later we will analyze the problem of stability in detail.

\subsection{Extending the Local Solution}

If we forget about the physical model we described with~\ref{eq:4.9} 
and take
$z_o < 0$, $t_o = 0$, for example, there will be a maximum time, 
$t_d = \frac1a \ln [\frac{a/b - z_o}{- z_o}]$,  
finite for which the solution diverges, that is,
\beq 
\lim_{t \rightarrow t_d} z(t) = \infty . 
\eeq
This shows that the previous theorem can only ensure the
existence of an interval $I$ for which the solution exists. The
most we can conclude in general, that is, without giving more information about $f$, is the following Corollary of
Theorem~\ref{teo:2}.

\begin{cor}
Let $U \subset  \re $ be the domain of definition of f, $U_c $ a
compact interval [that is, closed and bounded] of $U$, $z_o \in U_c $.
Then the solution $(\fip (z_o,t_o,t), I)$ of equation~\ref{eq:4.5} can be extended either indefinitely or up to
the boundary of $U_c $. 
This extension is unique in the sense that any pair of solutions,
$(\fip_1(z_o,t_o,t), I_1 )$, $\fip_2(z_o,t_o,t), I_2 )$ coincide in
the intersection of their definition intervals, $I = I_1 \cap
I_2$. 
\label{cor:4_2}
\end{cor}

\pru:
First, we prove the global version of the uniqueness of the solution.
Let $T$ be the smallest of the upper bounds of the set 
$\{\tau\, |\, \varphi_1(z_o,t) = \varphi_2(z_o,t) \,
for all \, t_o \le t \le \tau \}$.
Suppose in contradiction that $T$ is an interior point of 
$I_1 \cap I_2 $. 
By continuity we have,
$\varphi_1(z_o,T) = \varphi_2(z_o,T) $, but by the result of
Theorem~\ref{teo:2}, using initial conditions
$\varphi_1(z_o,T)$, at $ T$
we see that both solutions must 
coincide in a neighborhood of $T$, which contradicts that $T$ is the maximum.
Thus, $T$ must be an endpoint of one of the definition intervals, so the solutions coincide in 
$I_1 \cap I_2 \cap \{t | t \geq t_o\} $. 
The case $t \leq t_o $ is treated similarly, and therefore the
solutions coincide in all $I_1 \cap I_2 $. 

%\fig{4cm}{Global uniqueness.}
\begin{figure}[htbp]
  \begin{center}
    \resizebox{7cm}{!}{\myinput{Figure/m4_4}}
    \caption{Global uniqueness.}
    \label{fig:4_4}
  \end{center}
\end{figure}

Second, we construct the extension. The idea is that if two solutions coincide in $I_1 \cap I_2 $, then we can combine them and form one
in $I_1 \cup I_2 $, 
\beq 
\fit =\left\{ \barr{ll}  \fip_1(t) &for all t \in I_1\\
           \fip_2(t) &for all t \in I_2. 
         \earr\right.
\eeq

%\fig{4cm}{Extending the solution.}
\begin{figure}[htbp]
  \begin{center}
    \resizebox{7cm}{!}{\myinput{Figure/m4_5}}
    \caption{Extending the solution.}
    \label{fig:4_5}
  \end{center}
\end{figure}

Let $T$ be the smallest upper bound of $\{\tau | \fit $ exists and is 
contained in $U_c \; for all \, t_o \leq t \leq \tau \}$.
If $T$ $= \infty $ there is nothing to prove, so suppose $T$ $<
\infty $. We will prove that there exists $\fit $ defined for all $t_o \leq 
t \leq $ $T$ and such that $\varphi (T) $ is one of the endpoints of $U_c $.
Corollary~\ref{cor1} tells us that given any $\hat z \in U$, there exist
$\epsilon_{\hat z} > 0 $ and a neighborhood of $\hat z$, $V_{\hat z} \subset U$ such
that for any $z \in V_{\hat z} $ there exists a solution with
initial conditions $\varphi (t_o) = z$ defined for all t in
$\left| t - t_o \right| < \epsilon_{\hat z} $.
Since $U_c $ is compact, we can choose a finite number of points
${\hat z_i }$ such that $U_c \subset \bigcup V_{\hat z_i}$. 
Let $\epsilon > 0 $ be the minimum of the $\epsilon_{\hat z_i} $.
Since $T$ is the smallest upper bound, there exists $\tau $ between $T -
\epsilon $ and $T$ such that $\fit \in U_c \, for all \, t_o \leq t \leq \tau $, 
but since 
$\varphi(\tau ) \in U_c $ 
is in one of the $V_{\hat z_i } $, there exists
a solution $\hat{\fip}(t) $ defined for all $t$ in 
$\left| t - \tau \right|
< \epsilon $ 
with initial condition 
$\hat{\fip}(\tau) = \fip(\tau) $.
Using $\hat{\fip}(t) $ we can now extend $\varphi $, whose extension 
we will call $\tilde\varphi $, to the interval $t_o \leq t < \tau + \epsilon $.
But 
$\tilde\varphi (\theta) \in U_c \, for all \, t_o < \theta <T$ 
since 
$\tilde\varphi(\theta) = \varphi(\theta) \in U_c$, 
and otherwise $T$ would not
be a smallest upper bound. By continuity $\tilde\varphi(T)
\in U_c$ and as by definition of $T$ for any open interval, $I_T $,
containing $T$,  $\tilde\varphi[I_T] \not\in U_c $ we conclude that 
$\tilde\varphi(T) $ is an endpoint of $U_c$ \epru

\subsection{The Non-autonomous Case}

This is the case where instead of~\ref{eq:4.5} we have 
\beq 
\frac{d\dot z}{dt} = f(z,t).
\eeq
The geometric interpretation is the same, only now the {\sl
little lines} have an angle that also depends on the variable t,
therefore we expect that if we start at a point $(t_o,z_o)$ and 
trace a curve tangent to the little lines, we will also obtain a unique
solution. This is indeed the case, but its discussion will be 
included in the general theory of first-order autonomous systems
that we will state later.

Assuming certain conditions for $f(z,t) $ -- for example, that it is
Lipschitz with respect to both variables -- it can be seen that local
flows also exist, but these now also depend on $t_o $ and not just
on the difference between the initial and final time. We will denote
these flows as $g^t_{t_o}(z)$ and the semigroup property they satisfy is
$g^{t}_{t_1} \circ g^{t_1}_{t_o} (z) = g^t_{t_o} (z) $.

There is a special class of non-autonomous equations that can be reduced
to the already studied case. This is the class where $f(z,t)$ can be written as
the quotient of two functions, one of $z$ and another of $t$,
\beq 
f(z,t) = \frac{g(z)}{h(t)} .
\eeq
In this case, the geometric interpretation is that the parameter $t$
is not the most suitable to describe this system. This is remedied by choosing
another parameter $\tau $ given by, $\frac{d\tau}{dt} = \frac1{h(t)}$, or 
\beq 
\tau - \tau_o = \int^t_{t_o} \frac{dt}{h(t)}. 
\eeq
Once $t(\tau) $ is obtained [See Theorem~\ref{teo:2}], we must solve,
\beq 
\frac{dz}{d\tau} = g(z(\tau )), 
\eeq
[Using Theorem~\ref{teo:2} again], obtaining $\varphi(\tau) $.
The solution with respect to the original parameter is obtained 
by defining $\fit = \varphi(\tau(t)) $.

\section{Reduction to First-Order Systems}

\defi: The {\bf general ordinary differential equation of order} {\it m} 
is an equation of the form:
\begin{equation}
F(x,x^{(1)},\ldots,x^{(m)},t)\,=\,0     \label{eq:4.1}
\end{equation}
\noindent 

That is, an implicit equation of $x$, a map between 
$\tilde I\subset \re$ and $\tilde U\subset \re^{m \times n}$, (i.e., $x$ is a
vector of $n$ components), and its derivatives (where $x^{(i)} \equiv \frac{d^i x}{dt^i}]$ 
denotes the $i$-th derivative with respect to the independent parameter, $t$) 
up to order 
{\it n}~\footnote{In reality, to correctly define~\ref{eq:4.1}, the open sets $\tilde U^i$ for the $i$-th derivative with respect 
to $x$ where $F$ is defined must be given.}.

\defi: We will say that the $m$--times differentiable map $x(t): I \to
\re^n$ is {\bf a solution} of the previous equation if $F$
evaluated in this map is well-defined and identically zero over 
the entire interval $I$.

In this course, we will assume that~\ref{eq:4.1} can be solved for $x^m$,
that is, that~\ref{eq:4.1} is equivalent to the following equation:
\begin{equation}
x^{(n)}\,=\,f(x,x^{(1)},\ldots,x^{(n-1)},t)       \label{eq:4.2}
\end{equation}
\noindent in open sets $I\subset\tilde I \subset \re$\  and \ $U\subset
\tilde U \subset \re^m$

\ejem: The ODE $F(x,x^{(1)},t)=((x^{(1)})^2 -1)=0 $ 
implies one of the following ODEs in our sense:
\begin{equation}
\begin{array}{l}
   x^{(1)}\,=\,\,1 \\ 
   or \\
   x^{(1)}\,=\,-1
\end{array}
\end{equation}
\espa
If we know the values of $x$ and its derivatives up to the $(m-1)$-th
order at a point $t_0$, then equation~\ref{eq:4.2} allows us to
know the $m$-th derivative at that point. 
If we assume that $f$ is continuous, then this allows us to know
$x$ and its derivatives up to the $(m-1)$-th order at a point $t_0+\eps$
sufficiently close (the error incurred will be of the order
of $\eps \times $ {\it derivatives of $f$}), but then we can use
equation~\ref{eq:4.2} again and thus approximately know
the $n$-th derivative of $x$ at $t_0 + \eps$. Proceeding in this way,
we can achieve an {\it approximate solution} in a given interval.
At least intuitively, it would seem that by making the interval $\eps$ smaller and smaller,
we should obtain a solution. This is indeed
the case! And we will prove it later in the course. The important
thing for now is the fact that to obtain a solution, we must
give at a point $t_0$ the value of all the derivatives of the unknown
variable of order less than the one that determines the system, that is, the one
that appears on the left side of equation~\ref{eq:4.2}.



\espa
Introducing $ \tilde{u}^1 \equiv x$ , $\tilde{u}^2\equiv x^{(1)}$ , $\ldots$ ,
$\tilde{u}^n \equiv x^{(m-1)}$ , we can write~\ref{eq:4.2} in the form:
\begin{equation}
\begin{array}{lcl}
   \dot{\tilde{u}}^1&=&\tilde{u}^2 \\
   \dot{\tilde{u}}^2&=&\tilde{u}^3 \\
           &\vdots&  \\
\dot{\tilde{u}}^m&=&f(\tilde{u}^1,\tilde{u}^2,\ldots,\tilde{u}^m,t)\\
   \end{array}
 \end{equation}
\noindent 
where we have denoted with a dot over the function its derivative with
respect to the parameter $t$. 
Therefore,~\ref{eq:4.2} is equivalent to a first-order equation for the vector of vectors 
$\ve{\tilde{u}} \subset U\subset \re^{n \times m}$,

\beq 
\dot{\ve{\tilde{u}}}=\tilde{\ve{V}}(\ve{\tilde{u}},t)  \label{eq:4.3}
\eeq 

\noindent where $\tilde{\ve{V}}$ is also a vector in some
subset of $\re^{n \times m}$.
If $\tilde{\ve{V}}(\tilde{u},t)$ depends explicitly on $t$, then
we can add to
$\ve{\tilde{u}}$ another component, the $(n \times m +1)$-th with $\tilde{u}^{n \times m +1}=t$, 
and therefore we have that $\dot{\tilde{u}}^{n \times m + 1)} = 1$.
Equation~\ref{eq:4.3}, plus this
last equation, take the form,
\beq
\dot{\ve{u}}\,=\,\ve{V}(\ve{u})      \label{eq:4.4}
\eeq
where $\ve{u} = (\ve{\tilde{u}}, \tilde{u}^{n \times m+1})$ and
%\pagebreak

\beq
V^i(\ve{u}) =\left\{ \barr{ll}

  \tilde V^i(\ve{\tilde{u}},\tilde{u}^{n \times m+1}) &\mbox{if $1\leq i \leq m\times n$}\\
            1                   & \mbox{ if $i = n \times m + 1) $ ,}
                               \earr
                      \right.       
\eeq

\noindent a map between $U\subset \re^{n \times m+1}$ and
$\re^{n \times m + 1}$.



\noindent Thus, we arrive at the following theorem.

\begin{teo}[Reduction]
If the function $f$ of the system~\ref{eq:4.2} of $m$ ordinary differential 
equations of order $n$ is differentiable, then this system is equivalent 
to a system of $m\times n+1$ first-order ordinary differential equations 
that do not explicitly depend on the independent variable. That is, for each
solution of the system~\ref{eq:4.2}, we can construct a solution of the
system~\ref{eq:4.4} and vice versa.
\label{teo:4.1}
\end{teo}

\pru:
 Clearly, given a sufficiently differentiable solution of the system~\ref{eq:4.2},
we can write
with it a vector $\ve u$ and it will satisfy (because we constructed it this way)
the corresponding system~\ref{eq:4.4}. Therefore, every solution
of the original system is a solution of the associated first-order system.

To prove the inverse, that is, that every solution 
$\ve{u}(t)$ of the system~\ref{eq:4.4} gives rise to a 
solution of the system~\ref{eq:4.2}, it is only necessary to take repeated
derivatives of the first components, $x(t) = u^1(t)$, and use
the equations in the corresponding order until obtaining 
the original system satisfied by the nth derivative of $x(t)$
\epru


This theorem is important because it allows us to encompass the entire theory
of ODEs by studying equation~\ref{eq:4.4}, 
which, as we will see, has a clear geometric meaning. 
However, it should be noted that many times in physics
there appear very special subclasses of equations with particular properties
of great physical importance that the general theory does not contemplate.

\ejem: Mathematical Pendulum
$$ \ddot x=k\,x \,\,\,\,\,\,u^1=x\,,\,u^2=\dot x$$
\beq \frac d{dt}\left(\barr{c}
                 u^1 \\ u^2
                 \earr \right)=\left(\barr{cc}
                                0 & 1\\
                                k & 0
                                \earr\right)\left(\barr{c}
                                              u^1 \\ u^2
                                              \earr\right)
\eeq 

\espa
Note that if $u = \gamma(t): I \rightarrow U $ is a solution of~\ref{eq:4.4}
then $\forall s\in \re $ such that $t+s \in I$, $\gamma(t+s)$ is also
a solution. Indeed,
\beq
\left.\frac d{dt}\gamma(t+s)\right|_{t=t_o} =\left. \frac d{dt}\gamma(t) 
\right|_{t=t_o+s}= V(\gamma(t_o+s)) = \left. V(\gamma(t+s)) \right|_{t=t_o}.
\eeq
Due to this, from now on we will take $t_o$, the point where we give 
the initial condition, equal to zero.

\ejer:
If $u=\sin(t):\re \rightarrow [0,1]$, is a solution of~\ref{eq:4.4}
prove that $u=\cos(t)$ is also a solution.



\section{ODE Systems}

Do ODE systems have a geometric interpretation?
As we have seen, they have the generic form:
\beq
\frac {dx^i}{dt} = v^i(x^j)\;\;\;\;\;\;\;\;\;\;\; i=1,\ldots,n.
\eeq
Let $\gat$ be a curve between an interval $I\in \re$ and $M$ a manifold of
$n$ dimensions. Let $p=\gato$ be a point of $M$ and
$(U,\fip=(x^1,\ldots,x^n))$ a chart with $p\in U$. Then
$\fip\circ\gat=(x^1(t) ,\ldots,x^n(t))$ is a map between $I$ and an
open set of $\re^n$, and $\dip\frac{dx^i}{dt}$ are the components of the
tangent vector to $\gamma$ in the coordinate basis $\{\ve{x_i}\}$ at the point
$\gat$ of $M$. Therefore, $v^i(x^j)$ are the components of a
vector field $\ve v$ in the coordinate basis $\{\ve{x_i}\}$ at the point
$p\in M$ corresponding to $\fip(p)=(x^1,\ldots,x^n)$. Thus, we see that
ODE systems can be interpreted as a vector field $\ve{v}$ on a manifold $M$, and their solutions as curves $\gat $ 
such that their tangent at every point is $\ve{v}$. 
\espa
\noi
\ejem: The Physical Pendulum
\beq
\left\{\barr{rcl}
   \ddot\theta  & = & -\sin\theta \;\;\;\;\mbox{\o} \\
   \dot \theta & = & z  \\
   \dot z & = & -\sin\theta 
   \earr\right.
\eeq
\noi
Here the vector field is $z\dip\frac{\partial}{\partial\theta} -
\sin \theta \dip\frac{\partial}{\partial z}$,

%\fig{8cm}{The Physical Pendulum.}
\begin{figure}[htbp]
  \begin{center}
    \resizebox{7cm}{!}{\myinput{Figure/m4_6}}
    \caption{The Physical Pendulum.}
    \label{fig:4_6}
  \end{center}
\end{figure}

\noi
note that the diagram in the plane repeats every $2\pi$ along the $\theta$ axis.
In reality, the equation makes physical sense on a
cylinder, with $\theta$ being the angular variable, 
so the manifold is a cylinder. 
The {\bf integral curves}, i.e., the images of the maps $\gat$ that are
solutions,
are constructed by following the vectors so that they are
always tangent to the curves.


%\fig{8cm}{The Physical Pendulum Manifold.}
\begin{figure}[htbp]
  \begin{center}
    \resizebox{7cm}{!}{\myinput{Figure/m4_7}}
    \caption{The Physical Pendulum Manifold.}
    \label{fig:4_7}
  \end{center}
\end{figure}

From the example, it is clear that in general, if we give a point of $M$
and follow the vectors from it, we can find 
a solution $\gat$ that passes through this point. 
If we have two solutions $\gamma_1(t)$,
$\gamma_2(t)$, they cannot cross each other -- without being tangent -- at
points where the vector field is non-zero [since
otherwise, their tangents would give two different vectors at the point].
If we assume that the vector field is differentiable, then it can be
seen that different curves can never touch, and therefore the
solutions are unique.

Of course, as seen for example in figure \ref{fig:4_7}, 
the same curve can intersect itself (note that this can only happen
if the system is autonomous, as otherwise, the inclusion of
the temporal variable prevents any curve from intersecting itself). 
In such a case, it can be proven that this solution is
{\bf periodic}, meaning that there exists $T>0$ such that $\gamma(t+T)=\gat$. Note
in particular that this implies that $\gamma$ is defined for all
$t$ in $\re$.

\espa \noi
\bpro If $M$ is a compact manifold (example: a sphere
or a torus), it can be proven using Corollary \ref{cor:4_2} that all 
its phase curves are defined for all $t$. 
However, there are some that are not
closed, give an example.
\epro

\subsection{First Integrals}

Given a vector field $\ve{v}$ on $M$, if there exists $f:M\to\re$ \ non-constant 
such that $\ve{v}(f)=0$
on $M$, then we will say that $f$ is a {\bf first integral or constant
of motion} of the ODE system generated by $\ve{v}$.

It can be shown that if $n$ is the dimension of $M$, then
locally, around points where $\ve v \neq 0$,
there exist $n-1$ functionally independent first integrals.
However, it is easy to find vector fields that
do not have any first integrals.

\espa
\noi
\yaya{Examples--Problems}

\bpro 
The system
$$\left\{\barr{lcl}
        \dot x_1 & = &x_1 \\
        \dot x_2  &  = & x_2 
        \earr\right.
$$      
\noi has none, why? Hint: Draw the integral curves.
\epro

\bpro
Consider the system 
$$\left\{\barr{lcl}
        \dot x_1 & = &k_1 \\
        \dot x_2  &  = & k_2 
        \earr\right.
$$      
\noi on the torus (obtained by identifying the points $(x_1,x_2)$ and
$(x_1+n,x_2+m)$ of $\re $). For which values of $k_1\,,\;k_2 $
does a first integral exist and for which does it not? Hint: same as 1).
\epro


\bpro
The Hamilton equations of classical mechanics form
an ODE system,

\begin{eqnarray}
    \dot q_i & = & \derp{H}{p_i}  \\
    \dot p_i & = & -\derp{H}{q_i}
    \end{eqnarray}

\noi $i=1,\ldots,n$, where $H:\re^{2n}\to\re$ is the Hamiltonian function
or energy. Show that $H$ is a first integral
of this system.
\epro

 The previous examples show that there are ODE systems that
globally are not derivable from a variational principle.

Just as knowing the energy of a Hamiltonian system
is an invaluable aid in understanding the quantitative form of its
motion, knowing a first integral of a general ODE system also provides great help since we know that the
phase curves are restricted to their level surfaces ($f=$
constant), which allows us to effectively reduce the order of the
equation by one: we only have to consider the problem on the manifold
given by the points $f= constant.$

\subsection{Fundamental Theorem of ODE Systems}
\bteo

Let $\ve{v}$ be a continuously differentiable $(C^1)$ vector field in a
neighborhood of $p\in M$. Then,

 i) There exists a neighborhood $U_p$ of $p$ and an open set $I\subset\re$,
such that given any $q\in U_p\;(t_0\in I) $, there exists a differentiable curve
$\gamma_q(t):I\to M$ satisfying 
\beq 
\barr{rcl}
  \left.  \dip\derc{\gamma_q}{t}\right|_t & = &\left.\ve{v}\right|_{\gamma_q(t)}  \\
    \gamma_q(t_0) & = &q.
      \earr
\eeq

 ii) If $\gamma_q^1$ and $\gamma_q^2$ satisfy condition i), then
     $\gamma_q^1=\gamma_q^2$ in $I^1\cap I^2$.

iii) $\gamma_q(t)$ is also $C^1$ when considered as
a function from $I\times U_p\to M$. That is, it is continuously
differentiable with respect to the initial condition.

 iv) Let $U$ be a compact (bounded) region of $M$, $\ve{v}\in C^1$,
and $ p\in U$, then $\gamma_p(t)$ can be extended (forward and/or
backward) to the boundary of $U$. In particular, if $M$ is compact
and $\ve{v}\in C^1(M)$, then $\gamma_p(t)$ exists globally (for all 
$t \in \re$).

\eteo


We will prove this theorem later, after introducing the necessary
mathematical tools.




\subsection{Parameter Dependence, Variation Equation}

A very important consequence of the fundamental theorem in applications is the following:
\bcor
Let $\ve{v}_{\lambda}$ be a differentiable vector field in $U\subset M $ that depends
differentiably on a parameter $\lambda\in A \subset \re$. Then given
$p\in U$, $t_0\in I$, and $\lambda_0\in A$, there exist neighborhoods
$U_p\,,\;I_{t_0} $, and $A_{\lambda_0}$ such that for any triplet
$(q,t,\lambda) \in U_p\times I_{t_0}\times A_{\lambda_0}$, there exists a
unique integral curve of
$\ve{v}_{\lambda},\gamma_{\lambda}(t):I_{t_0}\times A_{\lambda_0}\to U_p$
with $\gamma_{\lambda}(0)=q$. This depends differentiably on $q$, $t$, 
and $\lambda$.
\ecor

\pru: Consider the vector field $(\ve{v}_{\lambda},0):U\times
A \to TM\times \re$. By hypothesis, this field is differentiable and therefore
has integral curves $(\gamma_{\lambda}\;,\; \lambda)$ that
depend differentiably on the initial conditions and therefore on $\lambda$ 
\epru

\espa
\noi\yaya{Warning}: The interval $I_{t_0}\times A_{\lambda_0}$ where
the solution is differentiable is not necessarily the interval of
definition of the solution. For example, it is not true that even
when the solution is defined for all $t$ in $\re$, the limits $\dip \lim_{T\to \infty}\gamma_{\lambda}(t)$ and $ \dip
\lim_{\lambda \to \lambda_0}\gamma_{\lambda}(t) $ commute.

The practical importance of this corollary is that it allows us to
obtain approximate solutions to already known ones. Indeed,
suppose that for $\lambda=0$ we know some integral curve of
$\ve{v}|_{\lambda=0}\,,\;\;\gamma_0(t)$ with $\gamma_0(0)=p$. Then,
applying the Corollary and a Taylor series expansion, we have 
 \beq 
 \gamma_{\lambda}(t)=\gamma_0(t)+\lambda\;\gamma_1(t)+O(\lambda^2)
 \eeq
 \noi that is, $\gamma_0(t)+\lambda\,\gamma_1(t)$ is a good
approximation to $\gamma_{\lambda}(t) $ for $ \lambda$ sufficiently
small.

It remains to find the equation that $\gay{1}{t}$ satisfies.
This is obtained by differentiating with respect to $\lam$ the differential
equation at $\lam =0$,
\beq
\der{\lambda}\left.\left(\derc{\gamma_{\lambda}(t)}{t}=\ve{v}_{\lambda}(\gay{\lambda}{t
})\right)\right|_{\lambda=0},
\eeq
expanding and using coordinates
\beq
\der{t}\gamma_1^j =\left. \derp{v^j}{x^i} \right|_{(\gamma_0,\lam=0)}\cdot
\gamma_1^i +\left. \derc{v^j}{\lambda}\right|_{(\gamma_0,\lambda=0)},
\eeq
$$
=A^j{}_{i}(t)\;\gamma_1^i +b^j(t).   \label{eq*}
$$
\noi that is, the equation for $\gay{1}{t}$ --usually called the variation equation-- is a linear inhomogeneous non-autonomous equation. Since
we took the initial condition $\gay{\lambda}{0}=p\;\;\;\forall\;\lam$, 
the initial condition we will consider for~\ref{eq*} is $\gay{1}{0}=0$.
\espa



\bpro 
What would be the initial condition for $\gamma_1$ if the
condition for $\gamma_{\lambda}$ were 
$ \gamma_{\lambda}(0)=p(1-\lam)+\lam q$? 
Consider only the one-dimensional case but think about the general case. 
\epro

\bpro
If we decided to consider an approximation up to the second
order $\gamma_{\lambda}=\gamma_0 +\lambda\gamma_1+\lambda^2\gamma_2+O(\lambda^3)$, what
equations would we obtain for $\gamma_2$?
\epro

\ejem: \textbf{a)} A body falls vertically in a low-viscosity medium, which depends on its position and velocity.
\beq
\ddot x = -g\;+\eps\;F(x,\dot x)\;\;\;\;\;\;\;\;\;\;\eps\ll 1
\eeq
\noi Calculate the effect of this resistance on the motion.

The solution, $x(t)$, will depend smoothly on $\eps$, so we can expand it in a Taylor series with respect to $\eps$.

\beq
x(t)= x_0(t) +\eps\;x_1(t)+O(\eps^2).
\eeq
\noi The solution for $\eps=0$ is clearly
\beq
x_0(t) = x_i+\ve{v}_it-g\frac{t^2}2 \;\;, 
\eeq
\noi where $x_i$ and $v_i$ are the initial position and velocity, respectively,
while the variation equation is 

\beq
\ddot x_1= \;F(x_0,\dot x_0),\;\;\;\;\;\;\;\;\;\;x_1(0)=0,\;\;\dot x_1(0)=0.
\eeq
\noi Integrating, we obtain
\beq
x_1(t)=\int_0^t \int_0^{\tau} F(x_0(\tilde\tau),\dot
x_0(\tilde\tau))\,\,d \tilde\tau\;d\tau.
\eeq
\vskip \baselineskip

\textbf{b)} The second example is self-oscillations.

\noi Consider the system,
\beq \barr{rcl}
       \dot x_1  &  = & x_2 + \eps\,f_1(x_1,x_2) \\
       \dot x_2  &  = & -x_1
                      +\eps\,f_2(x_1,x_2),\;\;\;\;\;\mbox{for}\;\;\;\eps\ll 1.
       \earr
\eeq
Without loss of generality, we can assume $f_i(0,0)=0$. When
$\eps=0$, we obtain the equations of the pendulum --in the small amplitude approximation--, that is, a conservative system. Its
energy --first integral-- is given by $E(x_1,x_2)=\frac12(x_1^2+x_2^2)$ and its
phase curves are then circles of radius $\sqrt{2E}$.

When $\eps>0$, the phase curves are not necessarily closed,
but due to the previous Corollary, they can at most be spirals with a
distance of the order of $\eps$ between turns. These can
tend towards some stationary point within the circle
$\sqrt{2E_i}$, where $E_i=$ initial energy, that is, where $ x_2 +
\eps\, f_1(x_1,x_2)=\;-x_1 \, +\;\eps\;f_2(x_1,x_2)=0$, or they can
asymptotically approach some closed orbit or finally
diverge. That
these are the only options is the result of a classic theorem by 
{\bf Poincaré--Bendixson}, which shows that in general these are the
only possibilities in two dimensions.

To study which case it is among these three, we consider the
variation of energy between two consecutive turns.

Using that,
\beq 
\dot E(x_1,x_2)\,=\;\eps\,(x_1\,f_1\;+\;x_2\,f_2)
\eeq
\noi we obtain
\beq 
\barr{rcl}
\Delta E\; &= &\;\eps\:\oint (-f_1\,dx_2\;+\;f_2\,dx_1)\;+\;O(\eps^2) \\
           &= & \;\eps\:F(E).
           \earr
\eeq
\noi where the integral is along a circle of radius
$\sqrt{2E}$ in the direction of motion. That is, we have
approximated the curve with ${\eps}>0$ by the curve with ${\eps}=0$.


If $F(E)$ is positive, the energy increases and the system undergoes
growing oscillations.

If $F(E)$ is negative, the oscillations decrease in amplitude and the
system eventually tends to an equilibrium point.

If $F(E)$ changes sign, say at $E=E_0$, it can be proven
that near this circle there is a closed phase curve
$\Gamma_{\eps}$. 

If $\left.F'(E)\right|_{E=E_0}$ is negative, $\Gamma$ is a stable limit cycle 
--any nearby phase curve asymptotically approaches it--. If $\left.F'(E)\right|_{E=E_0}$ is positive,
$\Gamma$ is unstable.

\bpro Show that if there are two stable limit cycles,
then there must also be at least one that is unstable.
\epro

From the two previous examples, the practical importance
of the Corollary is clear. The differentiable dependence on the
initial conditions also leads to the variation equation,
therefore this equation, which is linear and inhomogeneous, is of utmost
importance and in the next chapter, we will study it in more detail.

%%%%%%%%%%%%%%%%%%%%%%%%%%%%%%%%%%%%%%%%%%%%%%%%%%%%%%%%%%%%%%%%%%%%%%%%%%%%%

%\newpage
\section{Problems}

%%%%%%%%%%%%%%%%%%%%%%%%%%%%%%%%%%%%%%%%%%%%%%%%%%%%%%%%%%%%%%%%%%%%%%%%%%%%%

\bpro[Kiseliov]
Find all the solutions of 
\begin{equation}
  \label{eq:tresmedios}
  \frac{dx}{dt} = \frac{3}{2} x^{2/3}.
\end{equation}
%
Hint: there are infinitely many, and they are obtained from segments of some
particular solutions.
Graph some of these solutions.
\epro

\bpro[Kiseliov]
Apply the existence and uniqueness theorem to determine the regions
where the following equations have a unique solution:

a) $\dot{x} = x + 3 x^{1/3}$.

b) $\dot{x} = 1 - \cot x$.

c) $\dot{x} = \sqrt{1-x^2}$.
\epro

\bpro[Kiseliov]
Solve the following equations, in all cases give the
general solutions as a function of an initial value $x_0$
for the initial time $t_0$.

a) $t \dot{x} + x = \cos(t)$. (Use the first integral of the homogeneous part.)

b) $\dot{x} + 2x = e^t$. (Use variation of constants.)

c) $(1-t^2)\dot{x} + tx = 2t$. (First solve the homogeneous part and then
add a particular inhomogeneous solution.)

d) $x \dot{x} = t - 2t^3$.
\epro

\bpro[Kiseliov]
Solve the following equations by making a change of variable in
the independent variable (t).

a) $\dot{x}x = -t$. ($t = \cos(s)$.)

b) $\dot{x}t - x =0$.

c) $\dot{x} + e^{\dot{x}} = t$.
\epro

\bpro[Kiseliov]
Graph the isoclines (lines of equal slope in the $(t,x)$ plane)
and then trace the solutions of the following equations:

a) $\dot{x} = 2t - x$.

b) $\dot{x} = \sin(x+t)$.

c) $\dot{x} = x -t^2 + 2t -2$.

d) $\dot{x} = \frac{x-t}{x+t}$.
\epro

\bpro 
Solve the equation:

$\dot{x} = A(t)x + B(t)x^n$ Hint: Use the change of variable $y = x^{1-n}$.
\epro

\bpro 
Solve the equation
\begin{equation}
  \label{eq:osc}
  \frac{dx}{dt} = i\lambda x + A e^{i \omega t} \;\; \lambda, \; \omega \in \re
\end{equation}
%
and see how its real part behaves.
Examine the cases: 

a) $A=0$, 
b) ($A \neq 0, \;\; \lambda \neq \omega$) and
c) ($A \neq 0, \;\; \lambda = \omega$).
\epro

\bpro 
The equation of the physical pendulum.

a) Graph the vector field of the equation
\begin{equation}
  \label{eq:pendulo_fisico}
  \frac{d^2\theta}{dt^2} = -\sin(\theta), \;\;\;-\pi \leq \theta \leq \pi. 
\end{equation}

b) Graph some integral curves around 
$(\theta=0, z=\frac{d\theta}{dt} = 0)$.

c) Graph the integral curves that pass through the point
$\theta = \pm \pi, z=0)$. Infer that the time required 
to reach this point along these curves is infinite.

d) Graph the vector field along a line $z=z_0$.
Infer from this that the solutions can never exceed the speed acquired when passing through the point $\theta=0$. Hint: Conclude this first for the
region $0 \leq \theta \leq \pi,\;\;\; 0 \leq z \leq z_0$ and then use the symmetry of the solution.

e) Let $E(z,\theta) := \frac{z^2}{2} - \cos(\theta)$. See that 
$\frac{dE}{dt} = 0$. Use this quantity to analyze the behavior
of the solutions with initial data $(z_0,\theta_0=0)$. In particular,
determine which cross the $z=0$ axis and which do not.
\epro

\bpro 
Consider the equation:
\begin{equation}
  \frac{dz}{dt} = \left\{
  \begin{array}{ll}
    f(z) & |z| < 1 \\
    iz   & |z| \geq 1
  \end{array}
  \right.
\end{equation}
Where $f(z)$ is continuous with $iz$ at $|z|=1$.
Infer that no matter the form of $f(z)$, the solutions
never escape the region $\{|z(t)| \leq \max\{|z(0)|, 1\}\}$.
\epro

\bpro 
Consider a body affected by a central force, that is,
\begin{equation}
  m\frac{d^2\vec{x}}{dt^2} = f(r)\vec{x}\;\;\; r:= |\vec{x}|.
\end{equation}
%

a) Find an equivalent first-order system.

b) Verify that given any (constant) vector $\vec{c}$, then
   $F(\vec{x},\vec{p}) := \vec{c}\cdot(\vec{x}\wedge \vec{p})$,
   where $\vec{p} := \frac{d\vec{x}}{dt}$, is a first integral.
\epro

\bpro 
Consider the equation:
\begin{equation}
  \frac{d\vec{x}}{dt} = \vec{x} \wedge \vec{\omega}(\vec{x}).
\end{equation}
%

a) See that $R:= \vec{x}\cdot\vec{x}$ is a first integral.

b) Conclude that the solutions of this system exist for all time.
\epro


\bpro[Strichartz]
Show that 
\begin{equation}
  \label{eq:Bessel}
  J_k(t) = \sum_{j=0}^{\infty} \frac{(-1)^j(t/2)^{k+2j}}{j!(k+j)!},
\end{equation}
%
has an infinite radius of convergence and satisfies the Bessel equation,
$x''(t) + (1/t)x'(t) + (1-k^2/t^2)x(t) = 0$.
\epro



\bpro
Solve the system:
\begin{equation}
  \frac{d}{dt}\left(
    \begin{array}{c}
      x_1 \\ x_2
    \end{array}
  \right) = 
    \left(
      \begin{array}{c}
        -x_2 + \eps(x_1^2 + x_2^2) \\
        x_1
      \end{array}
    \right) 
\end{equation}
For the initial data $(x_1(0) = 1, x_2(0) = 0)$. Now investigate 
the solution near $\eps=0$ by obtaining the coefficients in its Taylor series 
with respect to the parameter $\eps$.
Find which equations these coefficients satisfy.
\epro

\bpro[Arnold]
Consider the equation of the physical pendulum: 
\[
\frac{d^2 x}{dt^2} = -\sin(x).
\]
See how the frequency varies as a function of the amplitude for 
small solutions. Hint: assume a solution in a Taylor series of
the amplitude and solve until finding the first non-trivial contribution
to the linearized solution. 
Find the period by locating the return points (zero velocity).
\epro

 
\recubib{For this and the next three chapters, I recommend the following books:
\cite{Arnold}, \cite{Braum}, and \cite{Roxin}. Especially the first one.
Ordinary differential equations are the basis of classical mechanics, although in the latter there is also a particular geometric structure
that is fundamental. Almost everything we have seen in this chapter is the {\sl local} theory, the {\sl global} theory, that is, the behavior of the solutions for large values of the independent variable (time in most cases) has only been understood in recent years, giving rise to new concepts such as chaos, attractors, fractal dimensions, etc. Unfortunately, these aspects have not yet been sufficiently synthesized, and therefore I cannot recommend any book on the subject. There are too many interesting ideas, but each of them requires a great deal of information.}


%%% Local Variables: 
%%% mode: latex
%%% TeX-master: "apu_tot"
%%% End: 