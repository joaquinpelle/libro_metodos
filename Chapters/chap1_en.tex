% !TEX encoding = IsoLatin9

%\input format

\chapter{Basic Concepts of Topology}

%\section{Topology}

\section{Introduction}

The notion of a set, while telling us that certain objects ---the elements that comprise it--- have something in common with each other, does not give us any idea of the {\it closeness} between these elements. On the other hand, if we consider for example the real numbers, this notion is present. We know, for example, that the number 2 is much closer to 1 than 423 is. The concept of a topology in a set that we will define below tries to capture precisely this notion of closeness which, as we will see, admits many gradations.

\defi: A {\bf topological space} consists of a pair $(X, \cT)$, where $X$ is a set and $\cT$ is a collection of subsets of $X$ satisfying the following conditions:

\begin{enumerate} \item The subsets $\emptyset$ and $X$ of $X$ are in \cT. \item If $O_{\lambda}$, $\lambda \in I$, is a one-parameter family of subsets of $X$ in \cT, then $\bigcup_I O_{\lambda}$ is also in \cT. \item If $O$ and $O'$ are in \cT, then $O \cap O'$ is also in \cT.

\end{enumerate}

The elements of \cT, subsets of $X$, are called the {\bf open subsets} of $X$. The set \cT itself is called a {\bf topology} on $X$. Condition 2) tells us that infinite unions of elements of \cT are also in \cT, while condition 3) tells us that in general only finite intersections remain in \cT. The following examples illustrate why this asymmetry exists, they also illustrate how giving a topology is essentially giving a notion of closeness between the points of the set in question.

\ejem: a) Let $\cT = \{\emptyset, X\}$, that is, the only open subsets of $X$ are the empty subset and the subset $X$. It is clear that this collection of subsets is a topology, as it satisfies the three required conditions. This topology is called the {\bf indiscrete topology} on $X$. We can say that, with this topology, the points of $X$ are arbitrarily close to each other, since if an open set contains one of them it contains all of them.

\ejem: b) Let \cT = \cP($X$), the collection of all subsets of $X$. Clearly, this collection also satisfies the conditions mentioned above and therefore it is also a topology on $X$, the so-called {\bf discrete topology on} $X$. We can say that in this one all the points are arbitrarily separated from each other since, for example, given any point of $X$ there is an open set that separates it from all the others, which consists of only the point in question.

\ejem: c) Let $X$ be the set of real numbers, \re, and let \\
$\cT = \{O \subseteq \re |\; \mathrm{if} \;r \in O, \, \exists \eps >0 \; \mathrm{such that if} \; |r - r'| < \eps, \; r' \in O}$, that is, the collection of open sets in the usual sense of the real numbers. Let's see that this collection satisfies the conditions to be a topology. Clearly, $\emptyset \in \cT$ (since it has no $r$), as well as \re$ \in \cT$ (since it contains all $r'$), and thus condition 1) is satisfied. Let us examine the second condition: let $r \in \bigcup_I O_{\lambda}$ then $r \in O_{\lambda}$ for some $\lambda$ and therefore there will exist $\eps >0 $ such that all $r'$ with $|r-r'| < \eps $ are also in $O_{\lambda}$, and therefore in $\bigcup_I O_{\lambda}$. 
Finally, the third condition: let $r \in O \cap O'$ then $r$ is in $O$ and therefore there will exist $\eps > 0$ such that all $r'$ with $|r-r'| < \eps $ will be in $O$; as $r$ is also in $O'$, there will exist $\eps'> 0$ such that all $r'$ with $|r-r'| < \eps'$ will be in $O'$. 
Let $\eps'' = \mathrm{min}\{\eps, \eps' \}$, then all $r'$ with $|r-r'| < \eps''$ will be in $O$ and in $O'$ and therefore in $O \cap O'$, so we conclude that this last set is also in \cT. $\re$ with this topology is called the {\bf real line}.

\ejer: In the real line, as defined in the previous example, find an infinite intersection of open sets that is not open.

\ejem: d) Let $X = \re \times \re \equiv \re^2$, that is, the Cartesian product of \re  with itself ---the set of all pairs $(x,y)$, with $x,y \in \re$--- and let \\
$\cT = \{O \in \reˆ2 |\; \mathrm{if} (x,y) \in O, \, \exists \eps >0 \mathrm{such that if} |x - x'|+|y-y'| < \eps, \; (x',y') \in O \}$. Following the previous example, it can be seen that this is also a topological space and that this is the topology we usually use in $\re^2$

\defi: A {\bf metric space} is a pair $(X,d)$ consisting of a set $X$ and a map $d: X \times X \to \re$, usually called \textbf{distance}, satisfying the following conditions for all $x, x', x'' \in X$:

\begin{enumerate} \item Non-negativity: $d(x,x') \geq 0$, with equality only when $x = x'$. \item Symmetry: $d(x,x')=d(x',x)$. \item Triangle inequality: $d(x,x') +d(x',x'') \geq d(x,x'')$. \end{enumerate}

\ejer: Prove that any metric space has a topology {\it induced} by its metric in a similar way to $\re$ in the previous example.

\ejer: Prove that, for any set, $d(x,y)=1$ if $x\neq y$ and $d(x,x)=0$ defines a distance. What topology does this distance induce on the set?

Clearly, a distance gives us a notion of closeness between points, in the precise form of a numerical value. A topology, by not generally giving us any number, gives us a much vaguer notion of closeness, but still generally interesting.

\subsection{Terminology}

We now give a summary of the usual terminology in this area, which is a direct generalization of the commonly used one.

\defi: We will call the {\bf complement}, $O^c$, of the subset $O$ of $X$ the subset of all elements of $X$ that are not in $O$.

\defi: We will say that a subset $O$ of $X$ is {\bf closed} if its complement $O^c$ is open.

\defi: A subset $N$ of $X$ is called a {\bf neighborhood of $x \in X$} if there is an open set $O_x$, with $x \in O_x$, contained in $N$.

\defi: We will call the {\bf interior} of $A \in X$ the subset $Int(A)$ of $X$ formed by the union of all open sets contained in $A$.

\defi: We will call the {\bf closure} of $A \in X$ the subset $Cl(A)$ of $X$ formed by the intersection of all closed sets containing $A$.

\defi: We will call the {\bf boundary} of $A \in X$ the subset $\partial A$ of $X$ formed by $Cl(A) - Int(A) \equiv Int(A)^c \cap Cl(A)$.

\ejer: Let $(X,d)$ be a metric vector space, prove that: \ a) $C^1_x = \{x'| d(x,x') \leq 1 \}$ is closed and is a neighborhood of $x$. \ b) $N^{\eps}_x = \{x' | d(x,x') < \eps \}$, $\eps >0$ is also a neighborhood of $x$. \ c) $Int(N^{\eps}_x) = N^{\eps}_x$ \ d) $Cl(N^{\eps}_x) = \{x' | d(x,x') \leq \eps \} $\ e) $\partial N^{\eps}_x = \{x' | d(x,x') = \eps \}$.

\ejer: Let $(X,\cT)$ be a topological space and $A$ a subset of $X$. Prove that: \ a) $A$ is open if and only if each $x \in A$ has a neighborhood contained in $A$. \ b) $A$ is closed if and only if each $x$ in $A^c$ (that is, not belonging to $A$) has a neighborhood that does not intersect $A$.

\ejer: Let $(X,\cT)$ be a topological space, let $A \in X$ and $x \in X$. Prove that: \ a) $x \in Int(A)$ if and only if $x$ has a neighborhood contained in $A$.\ b) $x \in Cl(A)$ if and only if every neighborhood of $x$ intersects $A$. \ c) $x \in \partial A$ if and only if every neighborhood of $x$ contains points in $A$ and points in $A^c$.

\section{Derived Concepts}

In the previous sections, we have seen that the concept of a topology leads us to a generalization of a series of ideas and derived concepts that we knew how to handle in $\re^n$, revealing that they did not depend on the usual distance used in these spaces (the so-called Euclidean distance). It is then worth asking if there are still other possible generalizations. In this and the next subsection, we will study two more of them. These in turn open up a vast area of mathematics, which we will not cover in this course but is very important in what concerns modern physics.

The first of these notions is \textit{continuity}.

\subsection{Continuous Maps}

\defi: Let $\fip :X \rightarrow Y$ be a map between two topological spaces (see the box below). We will say that the map $\fip$ is {\bf continuous} if given any open set $O$ of $Y$, $\fip^{-1}(O)$ is an open set of $X$.

\espa %%%%%%%%%%%%%%%%%%%%%%%%%%%%%%%%%%%%%%%%%%%%%%%%%%%%%%%%%%%%%%%%%%%%%%%%%

\recu{\defi: A {\bf map} $\phi:X \to Y$ between a set $X$ and another $Y$ is an assignment to {\it each} element of $X$ of an element of $Y$.

This generalizes the usual concept of a function, note that the map is defined for every element of $X$, while in general its {\bf image}, that is, the set $\phi(X) \equiv {y \in Y ;| ; \exists x \in X; y; \phi(x) = y}$, is not all of $Y$. In the case that it is, that is, $\phi(X) = Y$, we will say that the map is {\bf surjective}. On the other hand, if it is fulfilled that $\phi(x) = \phi(\ti x) \Longrightarrow x=\ti x$ we will say that the map is {\bf injective}. In such a case, there exists the inverse map to $\phi$ between the set $\phi(X) \subset Y$ and $X$. If the map is also surjective then its inverse is defined on all of $Y$ and in this case, it is denoted by $\phi^{-1}:Y \to X$. It is also useful to consider the sets $\phi^{-1}(O) = { x \in X ;|; \phi(x) \in O}$

} %%%%%%%%%%%%%%%%%%%%%%%%%%%%%%%%%%%%%%%%%%%%%%%%%%%%%%%%%%%%%%%%%%%%%%%%%%%%

\espa

Clearly, the previous definition only uses topological concepts. Does it have anything to do with the usual {\sl epsilon-delta} used in $\re^n$? The answer is affirmative, as we will see below in our first theorem, but first, let's see some examples.

\ejem: a) Let $X$ and $Y$ be any sets and let the topology of $X$ be the discrete one. Then any map between $X$ and $Y$ is continuous. Indeed, for any open set $O$ in $Y$, $\fip^{-1}(O)$ is some subset in $X$, but in the discrete topology, every subset of $X$ is an open set.

\ejem: b) Let $X$ and $Y$ be any sets and let the topology of $Y$ be the indiscrete one. Then any map between $X$ and $Y$ is also continuous. Indeed, the only open sets in $Y$ are $\emptyset$ and $Y$, but $\fip^{-1}(\emptyset) = \emptyset$, while $\fip^{-1}(Y)=X$, but whatever the topology of $X$, $\emptyset$ and $X$ are open sets.

From the previous examples, it might seem that our definition of continuity is not very interesting, that is because we have taken cases with the {\sl extreme} topologies, in the intermediate topologies is where the definition becomes more useful.

\ejem: c) Let $X$ and $Y$ be real lines, and let $\fip(x) = 1$ if $x \geq 0$, $\fip(x) = -1$ if $x < 0$. This map is not continuous because, for example, $\fip^{-1}((1/2,3/2)) = {x | x \geq 0}$.

\bteo The map $\fip:X \to Y$ is continuous if and only if it is fulfilled that: given any point $x \in X$ and any neighborhood $M$ of $\fip(x)$, there exists a neighborhood $N$ of $x$ such that $\fip(N) \subset M$. \eteo

This second definition is much closer to the intuitive concept of continuity.

\pru: Suppose $\fip$ is continuous. Let $x$ be a point of $X$, and $M$ a neighborhood of $\phi(x)$. Then there exists an open set $O$ in $Y$ contained in $M$ and containing $\phi(x)$. By continuity $N = \fip^{-1}(O)$ is an open set of $X$, and as it contains $x$, a neighborhood of $x$. It is then fulfilled that $\fip(N) \su O \su M$. Now suppose that given any point $x \in X$ and any neighborhood $M$ of $\fip(x)$, there exists a neighborhood $N$ of $x$ such that $\fip(N) \subset M$. Then let $O$ be any open set of $Y$, we must now show that $\fip^{-1}(O)$ is an open set of $X$. Let $x$ be any point of $\fip^{-1}(O)$, then $\fii (x) \in O$ and therefore $O$ is a neighborhood of $\fii (x)$, therefore there exists a neighborhood $N$ of $x$ such that $\fip(N) \su O$ and therefore $N \su \fip^{-1}(O)$. But then $\fip^{-1}(O)$ contains a neighborhood of each of its points and therefore it is open.

\ejer: Let $\phi : X \to Y$ and $\psi : Y \to Z$ be continuous maps, prove that $\psi \circ \phi : X \to Z$ is also continuous. (Composition of maps preserves continuity.)

\espa %%%%%%%%%%%%%%%%%%%%%%%%%%%%%%%%%%%%%%%%%%%%%%%%%%%%%%%%%%%%%%%%

\recu{{\bf \yaya{Induced Topology}:} \espa

Let $\phi$ be a map between a set $X$ and a topological space ${Y,\cT}$. This map naturally provides, that is, without the help of any other structure, a topology on $X$, denoted by $\cT_{\phi}$ and called the {\bf topology induced} by $\phi$ on $X$. The set of its open sets is given by: $\cT_{\phi} = {O \subset X ;|; O = \phi^{-1}(Q), ; Q \in \cT}$, that is, $O$ is an open set of $X$ if there exists an open set $Q$ of $Y$ such that $O = \phi^{-1}(Q)$.

\ejer: Prove that this construction really defines a topology.

Not all topologies thus induced are of interest and in general, they depend strongly on the map, as shown by the following example:

\ejem:

\noi a) Let $X=Y=\re$ with the usual topology and let $\phi: \re \to \re$ be the function $\phi(x) = 17$ $\forall ; x \in \re$. This function is clearly continuous with respect to the topologies of $X$ and $Y$, those of the real line. However, $\cT_{\phi}$, the topology induced on $X$ by this map is the indiscrete one!

\noi b) Let $X$ and $Y$ be as in a) and let $\phi(x)$ be an invertible map, then $\cT_{\phi}$ coincides with the topology of the real line.

} %%%%%%%%%%%%%%%%%%%%%%%%%%%%%%%%%%%%%%%%%%%%%%%%%%%%%%%%%%%%

\subsection{Compactness}

The other generalization corresponds to the concept of {\bf Compactness}.
For this, we introduce the following definition:
Let $X$ be a set, $A$ a subset of it, and $\{ A_{\lambda} \}$
a collection of subsets of $X$ parameterized by a continuous
or discrete variable $\lambda$. We say that this collection {\bf covers}
$A$ if $A \subset \cup_{\lambda}A_{\lambda}$.

\defi: We say that $A$ is {\bf compact} if given any collection
$\{A_{\lambda}\}$ of {\it open sets} that cover it, there exists a {\it finite}
number of these $A_{\lambda}$ that also cover it.

\ejem: a) Let $X$ be an infinite set of points with the discrete topology.
Then a covering of $X$ consists, for example, of all the
points of it, each considered as a subset of it. But the topology of $X$ is
discrete, so this is a covering of open sets and no finite number of them
will cover it, therefore $X$ is not compact in this case.
Clearly, if $X$ had only a finite number of elements
it would always be compact, regardless of its topology.

\ejem: b) Let $X$ be any set with the indiscrete topology.
Then $X$ is compact. The only open sets of this set are
$\emptyset$ and $X$, so any covering has $X$ as
one of its members and this alone is enough to cover $X$.

Thus, we see that this property strongly depends on the topology
of the set. The relationship with the intuitive concept of compactness
is clear from the following example and exercise.

\ejem: c) Let $X$ be the real line and $A = (0,1)$. This subset is not compact
because, for example, the following is a covering of open sets of $A$ such that
any finite subset of it is not. $A_n = (\frac{1}{n}, \frac{n-1}{n})$

\ejer: Let $X$ be the real line and $A = [0,1]$. Prove that $A$ is compact.

\bpru

Let $\{A_{\lambda}\}$ be a covering of $[0,1]$ and $a \in [0,1]$ the least upper bound 
of the $x \in (0,1]$ such that $[0,x]$ is covered by a finite subcovering. 
$a$ exists because $0$ has an $A$ that covers it. Let $A_{\lambda_0}$ be an element of the covering
such that $a \in A_{\lambda_0}$. Then there exists $b > a$ such that $b \in A_{\lambda_0}$ and $b$ is already 
covered by a finite subcovering. Thus, $a$ is in a finite subcovering and therefore,
if $a \neq 1$ also some elements greater than it. This would constitute a contradiction. 
\epru

Now let's see the relationship between the two derived concepts of
Topology, namely the continuity of maps between topological
spaces and compactness. The fact that a map between
topological spaces is continuous implies that this map is special,
in the sense that {\it it carries or conveys information about the respective
topologies and preserves the topological properties of the sets
it associates}. This is seen in the following property, which --as derived
from the following example-- is very important.

\bteo
Let $X$ and $Y$ be two topological spaces and $\phi$ a continuous
map between them. Then if $A$ is a compact subset of $X$, 
$\phi(A)$ is a compact subset of $Y$.
\eteo

\pru:
Let $O_{\lambda}$ be a collection of open sets in $Y$ that cover $\phi(A)$.
Then the collection $\phi^{-1}(O_{\lambda})$ covers $A$, but $A$ is
compact and therefore there will be a finite subcollection 
$\phi^{-1}(O_{\tilde{\lambda}})$
of the former that also covers it. Therefore, the finite subcollection
$O_{\tilde{\lambda}}$ will also cover $\phi(A)$. Since this is
true for any collection of open sets covering $\phi(A)$
we conclude that it is compact.

\ejem: Let $A$ be compact and let $\phi:A \to \re$ be continuous, that is, a map 
continuous between $A$ and the real line. $\phi(A)$ is then a compact set
in the real line and therefore a closed and bounded set, but
then this set will have a maximum and a minimum, that is, the
map $\phi$ reaches its maximum and minimum in $A$.


Finally, another theorem of fundamental importance about compact sets,
which shows that they have another property that makes them
very interesting. For this, we introduce the following definitions,
which also only use topological concepts.
A {\bf sequence} in a set $X$ 
$\{x_n\} = \{x_1, x_2, ...\}$, with $x_n \in X$, is a map
from the integers to this set.
Given a sequence $\{x_n\}$ in a topological space $X$, we say that 
$x \in X$ is a {\bf limit point} of this sequence if given
any open set $O$ of $X$ containing $x$ there exists a number $N$
such that for all $n > N$ $x_n \in O$. We say that 
$x \in X$ is an {\bf accumulation point} of this sequence if given 
any open set $O$ of $X$ containing $x$, infinitely many elements of the
sequence also belong to $O$. 

\espa

\ejer: Find an example of a sequence in some topological
space with different limit points. 

\espa
%\newpage

\bteo
Let $A$ be compact. Then every sequence in $A$ has an accumulation point.
\eteo

\espa

\pru:
Suppose, --contrary to the theorem's assertion-- that
there exists a sequence $\{x_n\}$ without any accumulation point.
That is, given any point $x$ of $A$ there exists a neighborhood $O_x$
containing it and a number $N_x$ such that if $n > N_x$ then $x_n \notin
O_x$. Since this is valid for any $x$ in $A$, the collection of
sets $\{ O_x | x \in A \}$ covers $A$, but $A$ is compact and therefore
there will exist a finite subcollection of these that also covers it.
Let $N$ be the maximum among the $N_x$ of this finite subcollection. But then
$x_n \notin A$ for all $n > N$ which is absurd.  

\ejer: Prove that compact sets in the real line are the
closed and bounded ones.

We can now ask the inverse question: If $A \subset X$ is
such that every sequence has accumulation points, is it true
then that $A$ is compact? An affirmative answer would give us
an alternative characterization of compactness, and this is
affirmative for the case of the real line. In general, the answer is
negative: there are topologies in which every sequence in a
set has accumulation points in it, but this set is not
compact. However, all the topologies we will see are {\bf second countable} [See box] and in these
the answer is affirmative.

In the real line, it is true that if $x \in \re $ is an accumulation point of a sequence $\{x_n\}$ then there exists a {\bf
subsequence}, $\{\tilde{x}_n\}$, that is, a restriction of the
map defining the sequence to an infinite number of natural
numbers, that will have $x$ as a limit point. This is also not
true in the generality of topological spaces, but it is if we consider only those that are {\bf first countable} [See box]. All the spaces we will see in this
course are.

\espa 
\vfill   %%%%%%%%%%%%%%%%%%%%%%%%%%%%%%%%%%%%%%%%%%%%%%%%%%%%%%%%%%%%%%%%%%%%%%

\recu{
{\bf *\yaya{Countability of topological spaces.}}
\espa

\defi: We say that a topological space $\{X,\cT\}$ is {first countable}
if for each $p \in X$ there exists a countable collection of open sets $\{O_n\}$ such that every open set containing $p$ also contains at least one of these $O_n$.

\espa

\defi: We say that a topological space $\{X,\cT\}$ is {second countable}
if there is a countable collection of open sets such that any open set of $X$ can be expressed 
as a union of sets from this collection.

\espa
\ejem:

\noi a) $X$ with the indiscrete topology is first countable.

\noi b) $X$ with the discrete topology is first countable. And second countable if and only if its elements are countable.

\espa
\ejer: Prove that the real line is first and second countable. Hint: For the
first case, use the open sets $O_n = (p - \frac{1}{n}, p + \frac{1}{n})$ and for the second 
$O_{pqn} = (\frac{p}{q} - \frac{1}{n}, \frac{p}{q} + \frac{1}{n})$

}  
%%%%%%%%%%%%%%%%%%%%%%%%%%%%%%%%%%%%%%%%%%%%%%%%%%%%%%%%%%%%%%%%%%%%%%
\espa

\recu{
{\bf *\yaya{Separability of topological spaces.}}
\espa

\defi: A topological space $X$ is Hausdorff if given any pair of points of $X$, $x$ and $y$,  there exist neighborhoods $O_x$ and $O_y$ such that $O_x \; \cap O_y = \emptyset$.

\ejem:

\noi a) $X$ with the indiscrete topology is not Hausdorff.

\noi b) $X$ with the discrete topology is Hausdorff.

\ejer: Find a topology such that the integers are Hausdorff and compact.

\ejer: Prove that if a space is Hausdorff then if a sequence has a limit point, this is unique.

\ejer: Let $X$ be compact, $Y$ Hausdorff, and $\phi :  X \to Y$ continuous. Prove that the images of closed sets are closed. Find a counterexample if $Y$ is not Hausdorff. 
}

\recubib{
This chapter is essentially a condensed version of chapters \textbf{26, 27, 28, 29,} and \textbf{30} of \cite{Geroch}, see also \cite{Kelley}, \cite{Wald}, and \cite{Isham}.
Topology is one of the most fascinating branches of mathematics, if you delve deeper you will be captivated! Of particular interest in physics is the notion of Homotopy, a good place to understand these ideas is chapter \textbf{34} of \cite{Geroch}.       }

\vfill
%\end{document}
%\end


%%% Local Variables: 
%%% mode: latex
%%% TeX-master: "apu_tot"
%%% End:
