% filepath: /Users/reula/Docencia/Metodos_libro/libro_metodos_github/Current_garamond/libro_gar.tex
% !TEX encoding = IsoLatin
% !TEX root =  ../Current_garamond/libro_gar.tex

%% last modification 28/05/2013

\chapter{The Fourier Transform}


\section{Introduction}

Consider the following problem on the circle $S^1$ whose
circumference is $2\pi$:

\noi Given $\rho$ continuous on $S^1$, find $f$ on $S^1$ such that 
 \beq
 \dersp{f}{\theta}=\rho. \label{10.1*}
 \eeq
 
\noi One way to solve this problem is by using the orthogonal
Fourier basis, 
$\lb\frac{1}{\sqrt{2\pi}}e^{in\theta}\rb$ of $L^2(S^1)$.
Let $F:L^2(S^1)\to l^2$ be the map between these spaces generated by this 
basis, that is,
\beq
F(f)=\lb\langle \frac1{\sqrt{2\pi}}e^{in\theta},f\rangle \rb\equiv\lb c_n\rb .
\eeq
Then 
\beq
\barr{rcl}
F\lp\dersp{f}{\theta}-\rho\rp &=& \lb\langle \frac1{\sqrt{2\pi}}e^{in\theta},
\dersc{f}{\theta}-\rho\rangle \rb \\ [3mm]
& = & \lb-n^2\,c_n -a_n\rb
\earr
\eeq
where $\{a_n\}=F(\rho)$.

Since $F(0)=\{0\}$ we see that \ron{10.1*} implies an algebraic 
equation in $l^2$,
\beq
-n^2c_n=a_n.\label{10.2*}
\eeq
If $\rho$ is $L^2$ orthogonal to $f=cte.$, that is to the only solutions of 
\ron{10.1*}
with $\rho=0$, which implies $a_0=0$, the sequence with $c_0$
arbitrary and
$c_n=-\dip\frac{a_n}{n^2}$, $n\neq0$ satisfies
\ron{10.2*}. 
The inverse map $F^{-1}:l^2\to L^2(S^1)$ defines 
$f(\theta)=\dip\sum_{n=-\ifi}^{\ifi}\frac{c_n}{\sqrt{2\pi}}e^{in\theta} $,
which at least formally\footnote{The map defines
an element of $L^2$ since $\rho\in L^2$ and therefore $\dip
\sum_{-\ifi}^{\ifi}|a_n|^2<\ifi$ which implies that $\{c_0$
arbitrary, $c_n=-\dip\frac{a_n}{n^2},\;\;n\neq0$\} is also 
an element of $l^2$.
It remains to be seen that $f=F^{-1}\{c_n\}$ is twice differentiable.}
is a solution of \ron{10.1*}.

Note that in this application we have not directly used the fact that
the functions $\{\dip\frac{1}{\sqrt{2\pi}}e^{in\theta}\}$
form an orthogonal basis but only that
$\dip\der{\theta}e^{in\theta}=in\,e^{in\theta}$ and certain properties
of the map $F$ generated by this basis. In particular, this property
of the basis is induced in the map in the sense that if $f\in L^2$ is
differentiable and $F(f)=\{c_n\}$ then
$F(\dip\derc{f}{\theta})=\{+in\,c_n\}$.

These observations are very useful since there are cases, such as $L^2(\re)$, 
in which interesting orthogonal bases are not known, but similar maps
to $F$ with interesting properties are known. 
One of them is the
Fourier transform that we will study now. 
The problem in $L^2(\re)$
is that we would like to have a basis $\{\fip_n\}$ whose functions 
satisfy $\dip\der{x}\fip_n=i\,c_n\fip_n$,
but the solutions to this equation are $\fip_n=a_n\,e^{ic_nx}$, 
which are not square-integrable functions for any $c_n$ or $a_n \;\in \; \re$  
(except of course if we take $a_n\equiv 0$). However, although
these functions do not form an orthogonal 
basis, they do generate a map $\cF{}$, this time from $L^2(\re)$ 
to another copy (considered as distinct) of $L^2(\re)$,
with properties similar to those of the map $F$
considered earlier.

\bteo[Fourier's Theorem] The Fourier transform 
\beq
\hat f(\lambda) := \cF{} (f)(\lambda) := \frac1{(2\pi)^{n/2}}\int_{\ren{}}
e^{-ix\cdot\lambda}\;f(x)\;d^nx 
\eeq
%
where  $x\cdot\lambda=\sum_{i=1}x^i\,\lambda_i$. \\

Satisfies: 
\espa

\noi a) $\cF{} :L^2(\ren{})\to L^2(\ren{})$
is a linear, continuous, and invertible map from $L^2(\ren{})$ onto itself that
preserves the norm (i.e., unitary),
$\|\hat f\|_{L^2}=\|\cF{}(f)\|_{L^2}=\|f\|_{L^2}$ (Plancherel's identity).

\espa
\noi b) Its inverse is given by
\beq
\check g(x)={\cal F}^{-1}(\hat
g)=\frac1{(2\pi)^{n/2}}\int_{\ren{}}e^{ix\cdot \lambda}\,\hat g(\lambda)\;d^n\lambda.
\eeq

\espa
\noi c) If the derivative of $f\in L^2(\ren{})$ in the distributional sense is
also in $L^2(\ren{})$, then
\beq
{\cal F}\lp\derp{f}{x_j}\rp(\lambda)=i\,\lambda_j\:{\cal F}(f)(\lambda).
\eeq
\eteo

This theorem tells us that this transform has all
the properties that the one generated by the Fourier basis had and
therefore will be as useful as that one in similar applications, but
now in \ren{}.

The proof of this theorem uses several of the most common and
powerful techniques of functional analysis and therefore a
careful and non-automatic reading of it is recommended.


\espa
\pru: 
The map \cF{} is obviously linear and clearly defined
for any function in $\cifrncs$, that is, infinitely differentiable
and compactly supported. First, let's see that \cF{} preserves the $L^2(\ren{})$ norm 
for these functions. 
Let $f\in\cifrncs$ be any function and take an n-cube $C_{\eps}$ of
volume $(\dip\frac{2}{\eps})^n$ with $\eps>0$ small enough 
so that $sop(f)\su C_{\eps}$.
Let $K_{\eps}=\{k\in\ren{} \;|\;k_i=\pi p\eps\mbox{ for some integer } 
p\}$, [for example in $\re^3$ the vector $(\pi\eps 5,\pi\eps17,0)$ is an
element of $K_{\eps}$]. 
Then the set of functions 
$\lb\lp\dip\frac{\eps}2\rp^{n/2}\,e^{ik\cdot x}\;|\;k\in K_{\eps}\rb$ 
forms an orthogonal basis
of $L^2(C_{\eps})$.

But $f\in L^2(C_{\eps})$ and is continuously differentiable, therefore 
{\small
\beq
\sum_{k\in K_{\eps}}\langle \lp\frac{\eps}2\rp^{n/2}\;e^{ik\cdot
\tilde{x}},f(\tilde{x} )\rangle \lp
\frac{\eps}2\rp^{n/2}\;e^{ik\cdot x}=\sum_{k\in K_{\eps}}\frac
{\hat f(k)}{(2\pi)^{n/2}}e^{ik\cdot x}(2\pi\eps/2)^n,
\label{10.f1}
\eeq
}
%
 is the Fourier series representation
of  $f$ which converges uniformly to $f$ in $C_{\eps}$ and therefore in \ren{}. 
We then have that 
\begin{eqnarray}
\|f\|^2_{L^2}&=&\int_{\ren{}}|f(x)|^2\;d^nx \nn 
&=&\int_{C_{\eps}}|f(x)|^2\;d^nx \nn 
&=&\sum_{k\in K_{\eps}}\lpi\langle (\frac{\eps}2)^{n/2}\,e^{ik\cdot x},f(x)\rangle \rpi^2 \nn
&=&\sum_{k\in K_{\eps}}|\hat f(k)|^2\;(\pi\eps)^n. 
\label{10.f2}
\end{eqnarray}
%
Since \ren{}{} is the union of n-cubes with sides $\pi\eps$ (i.e., of volume 
$(\pi\eps)^n$
around the points of $K_{\eps}$, the right-hand side of \ron{10.f2} is 
simply the
Riemann series of the function $|\hat f(k)|^2$. If this function were 
continuous
then the Riemann series would converge to $\|\hat f\|^2_{L^2}$ and 
we would have
proven that the Fourier transform of a function in $\cifrncs$
preserves the $L^2(\ren{})$ norm.
But 
\beq\barr{rcl}
sup_k|\derp{\hat f(k)}{k_j}| &=& sup_k|\frac1{(2\pi)^{n/2}}\dip\int_{R^n}
-i\,x^j\,e^{-ik\cdot x}\,f(x)\;d^nx|\\ [3mm]
&\leq&\frac1{(2\pi)^{n/2}}\dip\int_{R^n}|x^j\,f(x)|\;d^nx<\ifi,
\earr
\eeq
since $f\in\cifrncs$.
We then see that the derivatives of $\hat f(k)$ are bounded and therefore 
that $\hat f$ is continuous. This result not only completes the
proof that \cF{} preserves the norm but also, since
$\hat f(k)$
is continuous, that the Riemann series on the right-hand side of the
equation \ron{10.f1} converges to the integral and therefore that $\cF{}^{-1}
(\cF{}(f))=f(x)$ $\forall\;\;f\in\cifrncs$.

If $f\in\;\cifrncs$ then the identity in point c) is shown trivially. 
%
It only remains to extend these results to arbitrary $f$ in $L^2(\ren{})$,
but $\cifrncs$ is a dense subspace of $L^2(\ren{})$, that is, every element 
$f\in L^2(\ren{})$
can be obtained as the limit (with respect to the $L^2(\ren{})$ norm) of a 
sequence $\{f_n\}$ of functions in $\cifrncs$, this allows us to extend the
action of \cF{} to every element of $L^2(\ren{})$. 
Indeed, let $f\in L^2(\ren{})$ be arbitrary and
let $\{f_n\}\to f $ with $f_n \in\cifrncs$ then the sequence $\hat f_n(k)=\cF{}(f_n)$ 
satisfies,
\beq
\|\hat f_n(k)-\hat f_m(k)\|_{L^2(R^n)}=\|f_n(x)-f_m(x)\|_{L^2(\ren{})}.
\eeq
Since $\{f_n\}$ is convergent it is Cauchy and therefore the equality of norms 
implies that $\{\hat f_n\}$ is also
Cauchy in $L^2(\ren{})$, but $L^2(\ren{})$ is complete and therefore there exists a unique
$\hat f\in L^2(\ren{})$ such that $\{\hat f_n\}\to\hat f$.
We will extend the map \cF{} to \LR{} by defining $\cF{}(f) := \hat f$.
This extension is clearly linear and bounded, therefore it is
continuous. 
Using the same reasoning we extend $\cF{}^{-1}$ to all \LR{} and
see that $\cF{}^{-1}(\cF{}(f(x)))=f(x)$ $ \forall\;f(x)\in\LR$ 
which proves that \cF{} is invertible, that its inverse is
$\cF{}^{-1}$ and is continuous. The same argument shows that the formula in c)
is also valid for any $f$ such that its gradient is also
in \LR{}. This completes the proof of the theorem $\spadesuit$

Note that the strategy has been to first take a space ($\cifrncs$) where
all properties obviously hold, then take a space that
contains the former and where it is dense, see that the map is there
continuous (with respect to the notion of continuity of the larger
space) and finally take the extension that this continuity provides us.

Are there other possible extensions? The answer is yes and now we will see
one of them that will allow us to use the Fourier transform in
distributions. We will do it in the form of a problem divided into several
exercises.

\espa
\noi
\yaya{Schwartz Notation}: $I_+^n$ will denote the set of $n$-tuples of
non-negative integers 
$\alpha=<\alpha_1,\ldots,\alpha_n>$ 
and 
$|\alpha|=\sum_{i=1}^n\alpha_i$.
We define,  $x^{\alpha} := (x^1)^{\alpha_1} (x^2)^{\alpha_2} \cdots (x^n)^{\alpha_n}$
Similarly, we denote by 
$D^{\alpha}\equiv\dip\frac{\pa^{\alpha}}{\pa x^{\alpha}}\equiv\dip\frac{\pa^{|\alpha|}}
 {\pa (x^1)^{\alpha_1}\cdots\pa (x^n)^{\alpha_n}}$, 
 that is, $D^{\alpha}$ is the differential operator that takes $\alpha_1$ partial derivatives with respect to $x_1$, $\alpha_2$ partial derivatives with respect to $x_2$, etc. The degree of $D^{\alpha}$ is $|\alpha|$.

\espa
\defi: 
We will call the {\bf Schwartz space} $\cS(\ren{})$ 
the vector space
of infinitely differentiable functions that decay
asymptotically faster than the inverse of any polynomial,
that is, 
\beq
\|\fip\|_{p,q}\equiv
\barr[t]{c} {\dip\sum} \\ {|\ti{\alpha}|\leq p} \\
{|\ti\beta|\leq q}
\earr 
sup_{x\in\ren{}}|x^{\alpha}D^{\beta}\fip(x)|<\ifi
\label{10.f5}
\eeq
$\forall\;\alpha,\beta\in I_+^n$, with the following notion of convergence: 
we will say that $\{\fip_n\}$, $\fip_n \in \cS$ converges to $\fip\in\cS$ if given $\eps>0$
for each pair of non-negative integers, $p$, $q$, there exists $N_{p,q}$ such that 
if $n>N_{p,q}$ then $\|\fip-\fip_n\|_{p,q}<\eps$.

\espa
\noi
\yaya{Exercises}:

\noi
 1) How would you define the notion of a convergent sequence in
this space?

\noi
2) Show that if $\{\fip_n\}\to \fip$ then $\{\fip_n\}$ 
converges in the above sense.

\noi
3) Show that $\cD$ is strictly contained in $\cS$. 
Hint: Find $\fip\in\cS$ with $sop(\fip)=\ren{}$.
\espa

The properties of this space that interest us are the following:

\blem
With the corresponding notion of convergence, the
space $\cS(\ren{})$ is complete.
\elem

\blem 
\label{lem10.2}
The space $\cifrncs$ is dense in \cS, that is, any element 
$\fip\in\cS$
can be obtained as the limit of a convergent sequence with
each of its members in $\cifrncs$.
\elem

\section{Exercises and Definitions}

\ejer: Prove Lemma~\ref{lem10.2}.

\blem
\label{lem10.3}
$\cF{}:\cS\to\cS$ is continuous and invertible with a continuous inverse.
\elem

\ejer: 
Prove Lemma~\ref{lem10.3}. Hint: Prove that given integers $p$ and $q$
there exist $\ti{p}$ and $\ti{q}$  
and $c>0$ such that for all $\fip\in\cS$ it holds
\beq
\|\hat{\fip}\|_{p,q}\leq c\;\|\fip\|_{\ti{p},\ti{q}}.
\eeq

\ejer: 
Find $\cF{}(e^{-\alpha x^2/2})$.

\defi:
The dual space to \cS, $\cS'$ is called the space 
of {\bf tempered distributions}.

\ejer:
Show that $\cS'$ is strictly contained in $\cD'$. Hint: Find $f$ such that $T_f\in\cD'$ and not in $\cS'$.

How to extend \cF{} to $\cS'$? 
Using the Plancherel identity and the polarization identity\footnote{That is, $\langle x,y \rangle = \frac14 \{ ||x+y||^2 -
||x-y||^2 + i ||x-iy||^2 - i ||x+iy||^2 \}$ }
we have that $\langle \fip,\psi \rangle = \langle \hat{\fip},\hat\psi \rangle$.
But then 
\beq
T_{\hat\sigma}(\hat{\psi})=\int\hat\sigma\,\hat\psi\;dk =
\int \sigma\,\psi\;dx = T_{\sigma}(\psi) := 
\hat{T}_{\sigma}(\hat{\psi}).
\eeq
\noi
This leads us to define in general
\beq
\hat T(\hat{\fip}) := T({\fip}).
\eeq

\ejem: The Fourier transform of the Dirac delta.
%
From the previous definition we have that,

\[
\hat{\delta}_{a} (\hat{\phi}) = \phi(a),
\]
%
but 

\[
\phi(a) = \frac1{(2\pi)^{1/2}} \int_{-\infty}^{\infty} e^{ika} \hat{\phi}(k) dk,
\]
%
and therefore the Fourier transform of the Dirac delta is the regular distribution given by,

\[
\hat{\delta}_{a} = T_{e^{ika}}.
\]

\ejer:
Calculate $\cF{}(\del'_a)$.

%\ejer: Use the previous result to prove that
%\[
%\overline{\cF(\overline{\cF(f)})} = f
%\]

What other properties does the Fourier transform have?
Note that from the identity in point c) of the Fourier Theorem it follows
that if $x^{\alpha}D^{\beta}f\in L^2(\ren{})$ then $k^{\beta
}D^{\alpha}\hat f\in L^2(\ren{})$. This essentially tells us that 
differentiability
in $x$ is equivalent to decay in $k$ and vice versa. In particular
this tells us that the Fourier transform of a compactly supported function is infinitely differentiable.\footnote{
In fact, the previous argument tells us that if $sop(f)$ is
compact then $D^{\alpha}\hat f\in L^2(\ren{})\;\;\;\forall\;\alpha\in I_+^n$. 
As we will see later, this implies that it is
infinitely differentiable.}
Moreover, it can be proven that it is analytic.

Another important property of the Fourier transform is that
it maps the product of functions to the convolution of functions and vice versa.

\defi:
Let $f,g\in\cS(\ren{})$. The {\bf convolution} of $f$ with $g$ 
denoted $f*g$ is the
function, also in $\cS(\ren{})$, given by
\beq
(f*g)(y)=\int_{\ren{}}f(y-x)\:g(x)\;d^nx.
\eeq

\bteo

\noi a) For each $f\in\cS(\ren{})$, the linear map $g\to f*g$ from $\cS(\ren{})$ 
to $\cS(\ren{})$ is continuous.

\noi b) $\widehat{f\,g}=\frac1{(2\pi)^{n/2}}\:\hat f*\hat g$ \ \ \ and\
\ \ \  $\widehat{f*g}
=(2\pi)^{n/2}\,\hat f\hat g$.

\noi c) $f*g=g*f$ \ \ \ and\ \ \  $f*(g*h)=(f*g)*h.$
\eteo

\pru:
Once point $b)$ is proven, the others follow
trivially. Make sure of it! 
We will then prove $b)$.
As we have seen $\langle \fip,\psi \rangle = \langle \hat{\fip},\hat\psi \rangle \;\;\;\forall\;
\fip,\psi\in\cS(\ren{})$. 
Applying this identity to $e^{ik\cdot x}\bar
f(x)$ and $g(x)$ we obtain,
 
\[
\langle e^{ik\cdot x} \bar{f} ,g \rangle = 
\langle \widehat{e^{i k\cdot x}\bar f},\hat g \rangle ,
\] 
%
but
\beq
\langle e^{ik\cdot x}\bar f, g \rangle =\int_{\ren{}}e^{-i k\cdot x}\,f(x)\,g(x)\;d^{n}x=
(2\pi)^{n/2}\;\widehat{fg}(k)
\eeq
\noi 
and
\begin{eqnarray}
\langle \widehat{e^{i k\cdot x} \bar{f}},\hat g \rangle &=&\int_{\ren{}}\lp\overline{
(2\pi)^{-n/2}\int_{\ren{}}e^{-i\lambda\cdot x + i k\cdot x}\,\bar
f(x)\;d^nx}\rp\;\hat g(\lambda)\;d^n\lambda \nonumber \\
&=&\int_{\ren{}}\hat f(k-\lambda)\,\hat g(\lambda)\;d^n\lambda \nonumber \\
&=&\hat f*\hat g.
\end{eqnarray}
%
The other formula is obtained by applying $\cF{}^{-1}$ to the previous one $\spadesuit$

Another interesting property of the convolution is obtained by noting
that since $\cS(\ren{})$ is closed with respect to the operation of
convolution $(f,g\in\cS(\ren{})\Longrightarrow f*g\in\cS(\ren{}))$
we can define the convolution of a tempered distribution with
a function in $\cS(\ren{})$ as,
\beq
(T*f)(\fip) := T(\ti f*\fip)\;\;\;\mbox{ where } \ti f(x) := f(-x).
\label{10.t**}
\eeq
 
\noi But,

\beq
T_{[x]}(\ti f*\fip) = T_{[x]}(\int f(y - x)\phi(y) \; dy) = \int T_{[x]}(f(y-x))\phi(y) \; dy,
\eeq
%
where we have included a sub-expression in $T$ to indicate that the distribution acts on the test function as a function of $x$.
Since $f \in \cS(\ren{})$, $T_{[x]}(f(y-x)) :=g(y)$ gives us a finite value for each $y \in \ren{}$ considered as a parameter, and therefore a function $g(y)$, this function is in fact continuous.~\footnote{That is, for any $\{y_{n}\} \to y$ we have, $g(y_{n}) = T_{[x]}(f(y_{n}-x)) \to T_{[x]}(f(y-x)) = g(y)$, since $(f(y_{n}-x) \to f(y-x)$ in $\cS(\ren{})$ and distributions are continuous as maps from $\cS(\ren{})$ to $\re$.} Therefore the integral in the previous expression is well-defined and we conclude that,

\beq
\label{eqn:dist-conv-reg}
(T*f)(\fip) = T_{g}(\phi).
\eeq
%
That is, this convolution is a regular distribution!

% filepath: /Users/reula/Docencia/Metodos_libro/libro_metodos_github/translated_text.tex
\noi
\yaya{Exercise}: Apply \ron{10.t**} to $T=T_g$ for $g\in\cS(\ren{})$ and
see that this definition makes sense.

On the other hand, note that if $f$ is an integrable function and $g \in \cS(\ren{})$,
$f*g$ is infinitely differentiable! This follows from the fact that, as

\begin{eqnarray}
\frac{d(f*g)}{dx}(x) &=& \frac{d}{dx}\int_{-\infty}^{\infty} f(x-y)g(y) \;dy \nonumber \\
                             &=& \int_{-\infty}^{\infty} \frac{df}{dx}(x-y)g(y) \; dy \nonumber \\
                             &=& -\int_{-\infty}^{\infty} \frac{df}{dy}(x-y)g(y) \; dy \nonumber \\
                             &=& \int_{-\infty}^{\infty} f(x-y)\frac{dg}{dy}(y) \; dy,
\end{eqnarray}
%
all derivatives are well defined and bounded, and therefore such convolution is infinitely differentiable even if $f$ was only integrable.

\noi This interesting property can even be extended to distributions as seen in the following lemma.
\blem
If $f \in \cS(\ren{})$, and $T \in \cS'(\ren{})$, $T*f$ is equivalent to an infinitely differentiable function.
\elem

In fact, we have,

\beq
\barr{rcl}
(T*f)'(\phi) &=& (T*f)(-\phi') \\
                 &=& T(\tilde{f}*(-\phi')) \\
                 &=& T(\tilde{f}'*\phi) \\
                 &=& (T*f')(\phi). \\
\earr
\eeq                 

\ejer: Combine this result with the previous one, equation (\ref{eqn:dist-conv-reg}) and find the corresponding regular distribution.

%%%%%%%%%%%%%%%%%%%%%%%%%%%%%%%%%

\section{*Basic properties of Sobolev Spaces}

%%%%%%%%%%%%%%%%%%%%%%%%%%%%%%%%%

As an application of the Fourier transform
we will now see the basic properties of Sobolev spaces.
These will be used later when we study 
the theory of partial differential equations.
The first step will be a generalization of Sobolev spaces. As we saw, Sobolev spaces are Hilbert spaces
with a norm derived from the following inner product.
\beq
\langle f,g \rangle_{H^m}=\sum_{k=0}^m \sum_{i,j,p=1}^n \langle \underbrace{\pa_i\pa_j\ldots\pa_p}
_{ k\mbox{ times }}f,\,\pa_i\pa_j\ldots\pa_pg\rangle_{L^2},
\label{10.tr1}
\eeq
where $f$ and $g$ are functions defined in any open set $\Omega$ 
of \ren{}.
If $\Omega=\ren{}$ then from the properties of the Fourier transform 
we get
\beq\barr{rcl}
\langle f,g \rangle_{H^m}&=&\sum_{k=0}^m \sum_{i,j,p=1}^n \langle \lambda_i\lambda_j\ldots\lambda_p\hat
f(\lambda),\lambda_i\lambda_j\ldots \lambda_p,\hat g(\lambda)\rangle_{L^2}\\
&=&\langle \hat f(\lambda)\sqrt{\dip\sum_{k=0}^m|\lambda|^{2k}},\hat g(\lambda)
\sqrt{\dip\sum_{k=0}^m|\lambda|^{2k}}\rangle_{L^2},
\earr
\eeq
that is, the functions $f(x)\in H^m(\ren{})$ are those whose Fourier transform
$\hat f$ decays sufficiently fast so that
$\hat f(\lambda)\sqrt{\dip\sum_{k=0}^m|\lambda|^{2k}}$ is square integrable.

This suggests generalizing the spaces by allowing non-integer and even negative indices through the following inner product,
\beq
\langle f,g \rangle_{H^s}=\langle \hat f(\lambda)(1+|\lambda|^2)^{s/2},\hat g(\lambda)(1+|\lambda|^2)^{s/2}\rangle_{L^2}.
\eeq

\noi
\yaya{Exercise}: Show that this is an inner product.
\espa

Note that for $s=m$ instead of the polynomial $\dip\sum_{k+0}^m|\lambda|^{2k}$, 
we now have $(1+|\lambda|^2)^m$. Since
there are positive constants $c_1$ and $c_2$ such that 
\beq
c_1(1+|\lambda|^2)^m\leq\sum_{k=0}^m|\lambda|^{2k}\leq c_2(1+|\lambda|^2)^m\eeq
this change in the norm is 
trivial in the sense that these norms are equivalent. 
In fact, as we will see in a special case, even the 1
in the polynomial can be ignored and still obtain an equivalent norm.
The first property we will see is the following lemma:

\blem
If $s'>s$ then $H^{s'}\subset H^s$.
\elem
 
 \noi
\yaya{Proof}: Trivial $\spadesuit$

\espa

In particular, if $s>0$ then $H^s \su H^0=L^2$, if $s<0$ then $L^2\su H^s$. 
%
What are the elements of $H^s$ for negative $s$?

Given $g \in H^{-s}$ I can define the following map from $H^s$ to $\ve C$,
$$
\Psi_g(f) := \langle \hat g, \hat f \rangle_{H^0} = \langle \frac{\hat g}{(1+|\lambda|^2)^{s/2}}, \hat f (1+|\lambda|^2)^{s/2} \rangle_{H^0}.
$$
This map is linear, well defined $\forall \; f,g \in C^{\infty}_0$
and can be extended to all $H^s$ (and $H^{-s}$) by continuity, since,
$$
|\Psi_g(f)| \leq ||g||_{H^{-s}} ||f||_{H^s}.
$$
Therefore we have a map between $H^{-s}$ and the dual of $H^s$, which
preserves the norm, moreover, it can be proven that this map is an isomorphism
(that is, it is also invertible) between these spaces.\footnote{
By the Riesz Representation Theorem, each $\Psi :H^s \to \ve C$
can also be written as $\Psi(f) = (\tilde{g}, f)_{H^s}$,
$\tilde{g} \in H^s$. 
If we take $g =\cF{}^{-1}((1+|\lambda||^2)^s\cF{}(\tilde{g}))\in
H^{-s}$ then $\Psi(f) = \Psi_g(f)$.}
Therefore we can identify $H^{-s}$ with the dual of $H^s$.

\espa
\noi
\yaya{Exercise}: Using that $\cifrn$ is dense in $H^s(\ren{})$ show that 
the Dirac delta
is in $H^s(\ren{})$ for all $s<-n/2$.
\espa

Perhaps the most important property of $H^s(\ren{})$ is the following,

\blem[of Sobolev] 
If $m<s-n/2$ then $H^s(\ren{})\su C^m(\ren{})$.
\elem
\espa
\pru:
It is enough to prove it for $m=0$, the rest follows by induction.
It is also enough to prove the inequality.
$\|f\|_{C^0}<C\;\|f\|_{H^s},\;\;s>\frac n2 $ for some constant $C > 0$ 
independent of $f$ and for all $f\in\cifrn$, the
rest follows by the continuity of the norm. 
But
\beq
\barr{rcl}
\|f\|_{c^0}=sup_{x\in R^n}|f(x)|&=&sup_{x\in R^n}\frac1{(2\pi)^{n/2}} \lpi\dip\int_{R^n}
e^{i\lambda\cdot x}\,\hat f(\lambda)\;d^nx\rpi \\ [3mm]
&\leq&\frac1{(2\pi)^{n/2}}\dip\int_{R^n}|\hat f(\lambda)|\;d^n\lambda \\ [3mm]
&\leq&\frac1{(2\pi)^{n/2}}\dip\int_{R^n}\dip\frac{|\hat f(\lambda)|
(1+|\lambda|^2)^{s/2}}{(1+|\lambda|^2)^{s/2}}\;d^n\lambda \\ [3mm]
&\leq&\frac1{(2\pi)^{n/2}}\|\hat f(\lambda)(1+|\lambda|^2)^{s/2}\|_{L^2}\;\|
\frac{1}{(1+\|\lambda|^2)^{s/2}}\|_{L^2} \\ [3mm]
&\leq&C\;\|f\|_{H^s}.
\earr
\eeq
where we have used that $s>n/2$ to prove that 
$\|\dip\frac{1}{(1+|\lambda|^2)^{s/2}}\|_{L^2}<\ifi$ $\spadesuit$ 

This lemma tells us that if $f\in H^m(\ren{})$
for sufficiently large $m$ then $f$ can be identified with an
ordinary, continuous, and even differentiable function.

Let $f$ be a continuous function in $\re^n_+=\{x\in\ren{}\;\/\;x_n\geq0\}$ 
and let $\tau f$ be the restriction of that
function to the hyperplane $x_n=0$, that is, $(\tau f)(x_1,x_2,\ldots,x_{n-1})=
f(x_1,x_2,\ldots,x_n=0)$.
Clearly, $\tau$ is a linear map and if $f$ is continuous in $\re_+^n$ then 
$\tau f$ is
continuous in $\re^{n-1}=\{x\in\ren{}\;/\;x_n=0\}$.
Can this map be extended to more general functions?
The answer is the following lemma which also shows why
it is necessary to extend Sobolev spaces to non-integer indices.

% filepath: /Users/reula/Docencia/Metodos_libro/libro_metodos_github/translated_text.tex
\blem[Trace]
Let $m>0$ be an integer, then:

i) $\tau:H^m(\re^n_+)\to H^{m-1/2}(\re^{n-1})$ is continuous.

ii) it is surjective.
\elem

\espa
\pru:
By the same argument as in the previous lemma to prove $i)$
it is enough to prove that there exists $c>0$ such that
\beq
\|\tau f\|_{H^{1/2}(R^{n-1/2})} <C \,\|f\|_{H^1(R^n_+)} \;\;\;\;\forall
\;f\in C_0^{\ifi}(\re^n_+)
\eeq

Let $\hat f(\lambda',x_n)$ be the Fourier transform of $f(x)$ with respect to the
coordinates $(x_1,x_2,\ldots,x_{n-1})$ of $\re^n_+$, then 
\beq
|\hat f(\lambda',0)|^2=-2\,Re\,\int_0^{\ifi}\derp{\hat f}{x_n}(\lambda',t)
\overline{\hat f(\lambda',t)}\;dt.
\eeq
Multiplying both sides of this equality by $(1+|\lambda'|^2)^{1/2}$ 
and integrating with respect to $\lambda'$ we obtain,

\beq\barr{rcl}
\|\tau f\|_{H^{1/2}(R^{n-1})}&=&\dip\int_{R^{n-1}}|
(1+|\lambda'|^2)^{1/2}\hat{f}(\lambda',0)|^2\;d^{n-1}\lambda' \\ [3mm]
&=&2\lpi\dip\int_{R^{n-1}}\dip\int_0^{\ifi}(1+|\lambda'|^2)^{1/2}
\derp{\hat f(\lambda',t)}{x_n}\overline{\hat f(\lambda',t)}\;dt\,d^{n-1}\lambda'
\rpi\\ [3mm]
&\leq&2\lb\dip\int_{R^{n-1}}\dip\int_0^{\ifi}\lpi\derp{\hat f}{x_n}(\lambda',t)\rpi^2
\;dt\;d^{n-1}\lambda'\rb^{1/2}\times\\ [3mm]
&&\times\lb\dip\int_{R^{n-1}}\dip\int_0^{\ifi}(1+|\lambda'|^2)
\;|f(\lambda',t)|^2\;dt\;d^{n-1}\lambda'\rb^{1/2}.
\earr
\eeq
Using that $|2\,a\,b|\leq a^2+b^2$ and Plancherel's identity we obtain
\beq\barr{rcl}
\|\tau f\|^2_{H^{1/2}(R^{n-1})}&\leq&\int_{R^n_+}\lb|f(x)|^2+\sum_{k=0}^{n-1}
|\pa_kf|^2+|\pa_n f|^2\rb\;d^nx\\ [3mm]
&=&\|f\|_{H^1(R^n_+)}.
\earr
\eeq
To prove surjectivity we need to see that given $g\in H^{1/2}(\re^{n-1})$ 
there exists at least one
(in fact infinitely many) $f\in H^1(\re^n_+)$ such that $\tau f=g$. 
Again it is enough to define an
anti-trace $K$ on $\cifrn$ and prove that there exists $C>0$ such that
\beq
\|K\,g\|_{H^1(R^n_+)} < C\:\|g\|_{H^{1/2}(R^{n-1})}\;\;\;\;\;\;
\forall\;g\in\cifo.
\eeq
Let $K(g)=\cF{}^{-1}(e^{-(1+|\lambda'|^2)^{1/2}\,x^n}\cF{}(g))$ 
and noting that the argument of $\cF{}^{-1}$ is in $\cS(\re^{n-1})$ 
we leave the proof
of the above inequality as an exercise.

In applications, we will have to consider functions defined only
in open sets of \ren{}, $\Omega$ and their boundaries $\pa\Omega$.
We will assume that $\Omega$ is such that its boundary, $\pa\Omega=\bar{\Omega}-\Omega$, 
is a smooth manifold.

Let $H^m(\Omega)$ be the Sobolev space obtained by taking the integral in the
inner product simply over $\Omega$ and completing the space
$C^{\ifi}_0 (\Omega)$
with respect to its norm and let $H^m_0(\Omega)$ be the one obtained with the same norm but
completing the space $C^{\ifi}(\Omega)$. If $m=0$ or if $\Omega=\ren{}$ 
then these spaces
coincide. But if $m\geq1$ and $\pa\Omega$ is non-empty then they are 
different
(obviously $H^m_0\su H^m$) since as derivatives are controlled in the norm, the
sequences of compactly supported functions cannot converge to a
non-zero function on the boundary.
How will we extend the results obtained in \ren{} to $\Omega$?
The key lemma for this is the following.

\blem
Let $\gamma$ be the restriction of functions in \ren{} to functions in $\Omega$,
then $$\gamma:H^m(\ren{})\to H^m(\Omega)$$ is continuous and surjective.
\elem
\espa

\pru:
The continuity is clear, it only remains to see the surjectivity.
We will prove surjectivity for the case $\Omega=\{x\in\ren{},\;x_n>0\}$, 
$\pa\Omega=\{x\in \ren{}\;/\;x_n=0\}.$
Let $f\in H^m(\Omega)$ then by the previous Lemma $\tau(\pa^k_n f)\in
L^2$ for $k=0,1,\ldots,m-1$. We extend $f$ for $x_n$
negative as,
\beq
f(x)\equiv\lc\sum_{k=1}^{m-1}\frac{(-x_n)^k}{k!}\,\tau(\pa_n^k\,f)\rc\;\lp
1-e^{1/x_n^2}\rp,\;\;x_n<0
\eeq
The summation makes the derivatives of $f$ continuous at $x_n=0$ and the
exponential makes the $H^m$ norm bounded.
But this norm will only depend on the norms of $\tau (\pa_n^k f)$ 
which in turn are bounded by $\|f\|_{H^{m}(\Omega)}$ and therefore 
there will exist $c>0$ such that $\|f(x)\|_{H^m(R^n)}<C\;\|f\|_{H^m(\Omega)}$
$\;\forall\;f\in\cifo$.
The general case is technically more cumbersome, the basic idea
is to use charts that map regions containing part of the 
boundary of $\Omega$ into $\ren{}^+$ mapping boundary to boundary. In each one
of these charts we know how to prove the corresponding inequality.
The global inequality is proved by assuming, for the sake of contradiction, that
it does not hold $\spadesuit$

This lemma allows us to immediately generalize the previous lemmas.

\defi: We will say that $f\in H^s(\Omega)$ if it admits an extension
$\ti f$ to \ren{} such that $\ti f\in H^s(\ren{})$
[Note that this agrees with what was proved for positive integer $s$]. 

It is immediate then that if $s'>s$ then $H^{s'}(\Omega)\su H^s(\Omega)$, 
that if $m<s-n/2$ then $H^s(\Omega)\su C^m(\Omega)$ and that if
 $m>0$ integer $\tau:H^m(\Omega)\to H^{m-1/2}(\pa\Omega)$  
 is continuous and surjective.
 In the last statement we use $H^{m-1/2}(\pa\Omega)$ where in general 
$\pa\Omega\neq$ open in
 $\re^{n-1}$ and therefore does not fit the previous definition. In the
case where $\partial \Omega$ is compact we will say that 
$f\in H^{m-1/2}(\pa\Omega)$ if, given any sufficiently small open cover
$\{U_i\}$ of $\pa\Omega$ such that there exists $\fip_i$, such that $(U_i,\fip_i)$ 
is a chart of $\pa\Omega$ then $f\in H^{m-1/2}(U_i)\;\forall
\; i$.

\section{Weak Compactness and Compact Embeddings}

We conclude this chapter with two lemmas. The second of them, a consequence of the first,
is of great importance in the theory of partial differential equations.

\blem 
\label{lem10.9}
Let $\Gamma_d$ be a cube in $\ren$ with sides of length $d>0$.
If $u \in H^1(\Gamma_d)$ then,
\beq
\|u\|^2_{H^0(\Gamma_d)} \leq d^{-n}|\int_{\Gamma_d}u d^nx|^2 + 
\frac{nd^2}2 \sum_{j=1}^n \|\pa_j u\|^2_{H^0(\Gamma_d)}.
\eeq
\elem

\pru:
It is sufficient to consider $u \in C^1(\Gamma_d)$. 
For any $x$ and $y$ in $\Gamma_d$ we have,
\beq
u(y)-u(x) = \sum^n_{j=1} \int^{y^j}_{x^j} \pa_j u (y^1,...,y^{j-1},s,x^{j+1},
...,x^n) ds.
\eeq
Taking its square and using the Schwarz inequality we obtain,
\begin{eqnarray*}
 |u(x)|^2 + |u(y)|^2&-&2\Re(u(x) u(y)) \nn 
&\leq& nd\sum^n_{j=1} \int^{d/2}_{-d/2} |\pa_j u (y^1,\dots,y^{j-1},s,x^{j+1},\ldots,x^n)|^2 ds.
\end{eqnarray*}
%
Integrating with respect to all $x^j$ and $y^j$ we obtain, 
\beq
2\,d^n\:\|u\|^2_{H^0(\Gamma_d)}\leq 2\,\lpi\dip\int_{\Gamma_d}
u\;d^nx\rpi^2+n\,d^{n+2}\;\dip\sum_{j=1}^n\|\pa_j\,u\|^{2}_{H^0(
\Gamma_d)},
\eeq
from which
the desired inequality trivially follows $\spadesuit$

\blem
\label{lem10.10}
Let $\Omega \subset \ren$ be compact with smooth boundary.
If a sequence $\{u_p\}$ in $H^1_0(\Omega)$ is bounded, then there exists
a subsequence that converges (strongly) in $H^0(\Omega)$. 
That is, the natural map~\footnote{By natural map between two Banach spaces we mean the following: recall that the elements of each of these spaces are equivalence classes of Cauchy sequences, which we can assume consist of smooth elements of some dense space, for example $C^{\infty}(\Omega)$. The natural map is defined as the one that sends element by element a Cauchy sequence from one space to the other. Since the norm of the output space bounds the norm of the input space, the sequence obtained in the output space is also Cauchy and therefore determines an equivalence class there, hence an element of the space.} $H^1_0(\Omega) \rightarrow H^0(\Omega)$ is compact.
\elem

\pru: 
Let $k = sup\{\|u_p\|_{H^1_0(\Omega)}\}$. 
Since $\Omega$ is bounded we can enclose it in a cube $\Gamma_D$
and extend each $u_n$ as zero in $\Gamma_D - \Omega$. 
Let $\eps > 0$ and $M$ be large enough
such that $\frac{2nk^2D^2}{M^2} < \eps$.
Decomposing $\Gamma_D$ into $M^n$ cubes $\Gamma_d^j$ with $d = D/M$ and using
that since $\{u_n\}$ is also bounded in $H^0(\Omega)$, the weak compactness of balls in Hilbert spaces ensures that there exists a 
subsequence 
$\{\ti u_p\}$  and $u \in H^0(\Omega)$
such that it converges weakly to $u$. 
We then see that there exists an integer
$N$ such that if $p$ and $q >N$,
\beq
\lpi\int_{\Gamma_d^j} (\ti u_p -\ti u_q)\;d^nx\rpi^2 < \frac{\eps}2\lp
\frac DM\rp^{2n}\frac{1}{D^n}.
\eeq
If we apply lemma~\ron{lem10.9} to each $\Gamma_d^j$
and sum over $j$ we obtain that 
$\forall \;p$ and $q > N$,
\beq \barr{rcl} \|\ti u_p -\ti u_q\|^2_{H^0(\Omega)} 
&\leq& \lp\frac{D}{M}\rp^{-n} \lp\sum^{M^n}_{j=1}\frac{\eps}2 \lp\frac DM\rp^{2n} \frac{1}{D^{n}} \rp 
           + \frac n2 \lp\frac DM \rp^2(2k^2) \\ [3mm]
&\leq& \frac{\eps}2 + \frac{\eps}2\\ [3mm]
&\leq& \eps.
\earr 
\eeq                              
But then $\{\ti u_p\}$ is a Cauchy sequence in $H^0(\Omega)$ 
and therefore
converges strongly to $u$ in $H^0(\Omega)$ $\spadesuit$
\espa


\recubib{I recommend the books: \cite{Treves} and \cite{Reed}. Sobolev spaces allowed us to understand much of the theory of nonlinear equations and previously the theory of linear equations with non-smooth coefficients. They are not complex ideas, but very useful and essential for research in partial differential equations.}


%%% Local Variables: 
%%% mode: latex
%%% TeX-master: "apu_tot"
%%% End: 