\chapter{Preguntas Te�ricas de Ex�men}

\bpro
Sea $S$ un conjunto finito. Considere el espacio de todas las funciones de $S$ en $\re$. 
Vea que �ste es un espacio vectorial. Encuentre una base y diga cual es su dimensi�n.

\epro

\bpro
Pruebe que $\dim V = \dim V^{*}$ y que existe un mapa invertible natural entre $V$ y $V^{**}$. 

\epro


\bpro
Vea que $V/W$ es un espacio vectorial.

\epro

\bpro
Vea que en dimensi�n finita todas las normas son equivalentes.

\epro

\bpro
Sea $V$ un espacio normado y $V^{*}$ su dual. Como se define la norma inducida en $V^{*}$?
Vea que si $\ve{v} \in V$ y $\omega \in V^{*}$, luego $|\omega(v)|\leq ||\omega|| ||\ve{v}||$.

\epro

\bpro
Sean $A$ y $B$ dos tensores de tipo $(1,1)$ actuando sobre un espacio vectorial normado. 
Defina la norma inducida en estos operadores. Muestre que 
\[
|| A B || \leq ||A|| ||B||
\]

\epro

\bpro
Muestre usando la definici�n de determinante dada en el te�rico, que dado dos tensores de tipo $(1,1)$, $A$ y $B$, luego $\det(AB) = \det(A) \det(B)$.

\epro


\bpro
Vea que el operador  exponencial est� bien definido.

\epro

\bpro
Vea que dado un mapa lineal $A$, el conjunto de todos los mapas lineales de la forma $e^{tA}$, $t\in \re$,
forman un grupo.

\epro


\bpro
Pruebe que dado un espacio vectorial complejo, $V^{C}$ siempre existe un par autovector-autovalor.

\epro

\bpro
Vea que si $\{u_{i}\}$ son un conjunto de autovectores tales que $\lambda_{i} \neq \lambda_{j}$, luego estos son linealmente independientes.
\epro

\bpro
Pruebe el lema de Sch\"u{}r usando la noci�n de espacio cociente.

\epro

\bpro
Encuentre todas las posibles formas de Jordan de matrices $3\times 3$.

\epro

\bpro
Dado un operador lineal, $A: V \mapsto W$, estudie la definici�n de operador adjunto.
Encuentre la relaci�n entre las componentes de estos en t�rmino de una base cualquiera.
Vea que si estos operadores tienen normas y con ellas el mapa es acotado, luego se cumple que
\[
||A'|| = ||A||.
\]

\epro

\bpro
Vea que los autovectores de un operador autoadjunto son perpendiculares entre s� si sus autovalores son distintos. Vea que los autovalores son reales. Vea que todo operador autoadjunto es diagonalizable.

\epro

\bpro
Vea que si $A$ es autoadjunto luego $e^{iA}$ es unitario. Vea que el conjunto $e^{itA}$, $t\in \re$ es un grupo.

\epro

\bpro
Pruebe que las cartas definiendo la variedad Non-Hausdoff forman un atlas.

\epro

\bpro
Defina la diferenciabilidad de un mapa entre dos variedades infinitamente diferenciales.

\epro

\bpro
Vea que $\dim T_{p} = \dim M$.

\epro

\bpro
Dado un punto $p$ de $M$ y una curva $\gamma(t)$ que pasa por ese punto a $t=t_{0}$, encuentre las componentes del vector tangente a dicha curva en ese punto en una carta cualquiera.
Exprese la derivada a lo largo de $t$ de una funci�n definida en un entorno de $p$ en dicha base.

\epro

\bpro
Vea que la diferencia entre dos conexiones es un tensor.

\epro

\bpro
Encuentre el laplaciano de un vector en coordenadas polares en el plano.

\epro

\bpro
Enuncie y entienda geom�tricamente el teorema fundamental de las ecuaciones ordinarias.
Aplique a ejemplos simples.

\epro

\bpro
Pruebe el teorema de completamiento de espacios normados.
De ejemplos de espacios de Banach.

\epro

\bpro
Pruebe que $l^{2}$ es completo (y por lo tanto Hilbert).

\epro

\bpro
Pruebe que dado un subespacio cerrado $M$ de un espacio de Hilbert $H$, 
\[
H = M \oplus M^{\perp}.
\]
D� ejemplos de subespacios cerrados.
\epro

\bpro
Pruebe el teorema de Riez.

\epro

\bpro
Pruebe la relaci�n de Parseval y d� dos ejemplos de bases donde se cumple.

\epro

\bpro
Pruebe el teorema de la distribuci�n primitiva. �selo en dos ejemplos concretos.

\epro

\bpro
Como se define la transformada de Fourier de una distribuci�n? Encuentre en que espacio
de Sobolev se encuentra la delta de Dirac.  

\epro

\bpro
Entienda la prueba de  la transformada inversa de Fourier.

\epro

\bpro
Estudie la clasificaci�n de las ecuaciones en derivadas parciales y 
la noci�n de superficie caracter�stica. 
Se dar�n ejemplos de ecuaciones para clasificar. 

\epro

\bpro
Pruebe el teorema de Poincar�--Hardy y diga como se emplea en la prueba de existencia del 
problema de Dirichlet para el Laplaciano. Entienda las generalizaciones de este teorema.

\epro

\bpro
Use y entienda el teorema del m�ximo para probar la unicidad del problema de Dirichlet.

\epro


\bpro
Enuncie y entienda las generalizaciones del teorema espectral.
D� ejemplos de aplicabilidad.

\epro

\bpro
Entienda las soluciones generales de la ecuaci�n de onda en 1 y 3 dimensiones. 
Entienda los conceptos de dominio de dependencia y de influencia.


\epro


\bpro
Use el teorema del m�ximo para probar la unicidad del problema de valores iniciales y contorno para la ecuaci�n del calor. 

\epro


\bpro
Encuentre todos los subgrupos de $P(1,2,3)$. Cuales de ellos son normales?
\epro



\bpro
Pruebe que si $N$ es subgrupo normal de $G$, luego $G/N$ es un subgrupo.
Muestre un ejemplo de esta construcci�n.
\epro



\bpro
Encuentre un subgrupo de $U(2)$. Ayuda, use el mapa exponencial. 
\epro








